\documentclass[12pt,a4paper]{book}
\usepackage[utf8]{inputenc}
\usepackage[spanish]{babel}
\usepackage{amsmath}
\usepackage{amsfonts}
\usepackage{amssymb}
\usepackage{graphicx}
\usepackage{geometry}
\usepackage{fancyhdr}
\usepackage{array}
\usepackage{booktabs}
\usepackage{float}
\usepackage{xcolor}
\usepackage{tikz}
\usepackage{pgfplots}
\usepackage{tcolorbox}
\usepackage{fontawesome5}
\usepackage{enumitem}
\usepackage{mdframed}
\usepackage{shadow}
\usepackage{multicol}
\usepackage{colortbl}

\geometry{left=2.5cm,right=2.5cm,top=3cm,bottom=3cm}
\setlength{\headheight}{30pt}
\pgfplotsset{compat=1.18}

% Definir paleta de colores moderna y atractiva
\definecolor{azulprincipal}{RGB}{41, 98, 255}
\definecolor{azulsecundario}{RGB}{0, 123, 255}
\definecolor{verdeprincipal}{RGB}{40, 167, 69}
\definecolor{verdesecundario}{RGB}{25, 135, 84}
\definecolor{naranjaacento}{RGB}{255, 193, 7}
\definecolor{rojoacento}{RGB}{220, 53, 69}
\definecolor{moradoacento}{RGB}{111, 66, 193}
\definecolor{grisclaro}{RGB}{255, 248, 220}
\definecolor{grisoScuro}{RGB}{108, 117, 125}
\definecolor{azulclaro}{RGB}{179, 229, 252}
\definecolor{verdeclaro}{RGB}{212, 237, 218}
\definecolor{naranjaclaro}{RGB}{255, 243, 205}
\definecolor{violetaclaro}{RGB}{232, 222, 248}
\definecolor{crema}{RGB}{255, 248, 220}

% Configuraciones para cajas coloridas
\tcbuselibrary{skins,breakable}

% Configuración de encabezados
\pagestyle{fancy}
\fancyhf{}
\fancyhead[LE,RO]{\colorbox{azulprincipal}{\textcolor{white}{\bfseries\thepage}}}
\fancyhead[LO,RE]{\textcolor{azulprincipal}{\bfseries Capítulo: Inventarios}}
\renewcommand{\headrulewidth}{2pt}
\renewcommand{\headrule}{\hbox to\headwidth{\color{azulprincipal}\leaders\hrule height \headrulewidth\hfill}}

\usetikzlibrary{shapes.geometric, arrows, shadows, patterns}

\begin{document}
	
	% PORTADA SUPER ATRACTIVA
	\begin{titlepage}
		\begin{tikzpicture}[remember picture,overlay]
			% Fondo color crema
			\fill[crema] (current page.south west) rectangle (current page.north east);
			\fill[azulsecundario,opacity=0.7] (current page.south west) rectangle ([yshift=8cm]current page.south east);
			\fill[verdeprincipal,opacity=0.5] (current page.south west) rectangle ([yshift=4cm]current page.south east);
			
			% Elementos decorativos
			\foreach \i in {1,...,20}
			{
				\fill[white,opacity=0.1] ([xshift=\i cm, yshift=\i cm]current page.south west) circle (0.5);
			}
			
			% Líneas decorativas
			\draw[white,line width=3pt,opacity=0.8] ([yshift=2cm]current page.south west) -- ([yshift=2cm]current page.south east);
			\draw[naranjaacento,line width=2pt] ([yshift=2.2cm]current page.south west) -- ([yshift=2.2cm]current page.south east);
		\end{tikzpicture}
		
		\centering
		\vspace{1.5cm}
		
		% Universidad
		\begin{tcolorbox}[colback=white,colframe=azulprincipal,boxrule=3pt,arc=15pt,drop shadow]
			\centering
			{\Large\bfseries\color{azulprincipal} \faUniversity\ UNIVERSIDAD NACIONAL DEL ALTIPLANO}\\[0.3cm]
			{\large\color{grisoScuro} Ingeniería Estadística e Informática}
		\end{tcolorbox}
		
		\vspace{1.5cm}
		
		% Título del capítulo con efectos
		\begin{tcolorbox}[colback=naranjaclaro,colframe=naranjaacento,boxrule=3pt,arc=15pt,drop shadow]
			\centering
			{\Huge\bfseries\color{rojoacento} CAPÍTULO: INVENTARIOS}\\[0.5cm]
			{\Large\color{grisoScuro} \faBoxes\ Modelos de Gestión Óptima}
		\end{tcolorbox}
		
		\vspace{1.5cm}
		
		% Iconos temáticos
		\begin{tikzpicture}
			\node[azulprincipal,scale=3] at (-3,0) {\faCalculator};
			\node[verdeprincipal,scale=3] at (0,0) {\faChartArea};
			\node[rojoacento,scale=3] at (3,0) {\faCogs};
		\end{tikzpicture}
		
		\vspace{1.5cm}
		
		% Subtítulos con iconos
		\begin{tcolorbox}[colback=violetaclaro,colframe=moradoacento,boxrule=2pt,arc=10pt]
			\centering
			{\large\bfseries\color{moradoacento} \faLightbulb\ EOQ • Descuentos • Probabilísticos • Optimización}\\[0.3cm]
			{\normalsize\color{grisoScuro} Modelos Determinísticos y Probabilísticos para la Gestión Empresarial}
		\end{tcolorbox}
		
		\vfill
		
		% Información del autor
		\begin{tcolorbox}[colback=grisclaro,colframe=grisoScuro,boxrule=2pt,arc=8pt,drop shadow]
			\centering
			{\Large\bfseries\color{azulprincipal} \faUser\ AUTOR}\\[0.5cm]
			{\LARGE\bfseries\color{rojoacento} Etzel Yuliza Peralta López}\\[0.3cm]
			{\large\color{grisoScuro} Material de Estudio Especializado}
		\end{tcolorbox}
		
	\end{titlepage}
	
	\newpage
	
	% ÍNDICE COLORIDO
	\begin{tcolorbox}[colback=azulclaro,colframe=azulprincipal,boxrule=2pt,arc=10pt,title={\Large\bfseries\color{white} \faList\ CONTENIDO DEL CAPÍTULO}]
		\tableofcontents
	\end{tcolorbox}
	
	\newpage
	
	% INTRODUCCIÓN CON DISEÑO ATRACTIVO
	\chapter{Inventarios}
	
	\begin{tcolorbox}[colback=naranjaclaro,colframe=naranjaacento,boxrule=2pt,arc=10pt,title={\large\bfseries\color{white} \faInfoCircle\ INTRODUCCIÓN}]
		
		Los modelos de inventarios constituyen una herramienta fundamental en la gestión empresarial moderna, permitiendo a las organizaciones optimizar sus recursos financieros y operativos. El inventario representa uno de los activos más significativos en muchas empresas, y su gestión eficiente puede determinar el éxito o fracaso de una organización.
		
	\end{tcolorbox}
	
	\section{Importancia de la Gestión de Inventarios}
	
	\begin{multicols}{2}
		\begin{tcolorbox}[colback=verdeclaro,colframe=verdeprincipal,boxrule=1pt,arc=5pt]
			\textbf{\faCheckCircle\ Minimizar costos totales:} Incluyendo costos de pedido, almacenamiento y escasez
		\end{tcolorbox}
		
		\begin{tcolorbox}[colback=azulclaro,colframe=azulprincipal,boxrule=1pt,arc=5pt]
			\textbf{\faUsers\ Mantener niveles de servicio:} Satisfacer la demanda del cliente
		\end{tcolorbox}
		
		\begin{tcolorbox}[colback=violetaclaro,colframe=moradoacento,boxrule=1pt,arc=5pt]
			\textbf{\faDollarSign\ Optimizar el capital de trabajo:} Evitar inversiones excesivas en inventario
		\end{tcolorbox}
		
		\begin{tcolorbox}[colback=naranjaclaro,colframe=naranjaacento,boxrule=1pt,arc=5pt]
			\textbf{Reducir riesgos:} Minimizar obsolescencia y deterioro
		\end{tcolorbox}
	\end{multicols}
	
	\subsection{Tipos de Modelos de Inventarios}
	
	\begin{tcolorbox}[enhanced,colback=grisclaro,colframe=azulprincipal,boxrule=2pt,arc=8pt,
		drop shadow,title={\bfseries\color{white} \faChartPie\ CLASIFICACIÓN POR PATRÓN DE DEMANDA}]
		
		\begin{itemize}[leftmargin=*,label=\textcolor{verdeprincipal}{\faArrowRight}]
			\item \textbf{\color{azulprincipal}Determinísticos:} La demanda es conocida y constante
			\item \textbf{\color{rojoacento}Probabilísticos:} La demanda es variable y sigue una distribución de probabilidad
		\end{itemize}
		
	\end{tcolorbox}
	
	\begin{tcolorbox}[enhanced,colback=verdeclaro,colframe=verdeprincipal,boxrule=2pt,arc=8pt,
		drop shadow,title={\bfseries\color{white} \faClock\ CLASIFICACIÓN POR TIEMPO DE ENTREGA}]
		
		\begin{itemize}[leftmargin=*,label=\textcolor{naranjaacento}{\faArrowRight}]
			\item \textbf{\color{verdeprincipal}Tiempo de entrega cero:} El pedido se recibe inmediatamente
			\item \textbf{\color{azulprincipal}Tiempo de entrega constante:} El tiempo entre pedido y entrega es fijo
			\item \textbf{\color{rojoacento}Tiempo de entrega variable:} El tiempo de entrega varía aleatoriamente
		\end{itemize}
		
	\end{tcolorbox}
	
	\section{Modelo EOQ Básico}
	
	\begin{tcolorbox}[enhanced,colback=azulclaro,colframe=azulprincipal,boxrule=3pt,arc=12pt,
		drop shadow,title={\Large\bfseries\color{white} \faLightbulb\ FUNDAMENTOS TEÓRICOS}]
		
		El modelo de Cantidad Económica de Pedido (EOQ) es el modelo más fundamental en la teoría de inventarios. Fue desarrollado por Ford W. Harris en 1913 y busca determinar la cantidad óptima de pedido que minimiza los costos totales de inventario.
		
	\end{tcolorbox}
	
	\subsection{Supuestos del Modelo EOQ}
	
	\begin{enumerate}[leftmargin=*,label=\textcolor{azulprincipal}{\faCircle\ \arabic*}]
		\item \textcolor{verdeprincipal}{\textbf{La demanda es determinística y constante}}
		\item \textcolor{azulprincipal}{\textbf{El tiempo de entrega es constante}}
		\item \textcolor{rojoacento}{\textbf{No se permiten faltantes}}
		\item \textcolor{moradoacento}{\textbf{El costo unitario es constante}}
		\item \textcolor{naranjaacento}{\textbf{Los costos de pedido y almacenamiento son constantes}}
		\item \textcolor{grisoScuro}{\textbf{La tasa de consumo es mayor que la de producción}}
	\end{enumerate}
	
	\subsection{Componentes de Costo}
	
	\begin{multicols}{3}
		\begin{tcolorbox}[colback=verdeclaro,colframe=verdeprincipal,boxrule=2pt,arc=8pt,center title,
			title={\bfseries\color{white} \faShoppingCart\ PEDIDO}]
			\centering
			Costo fijo incurrido cada vez que se realiza un pedido, independiente de la cantidad ordenada.
		\end{tcolorbox}
		
		\begin{tcolorbox}[colback=azulclaro,colframe=azulprincipal,boxrule=2pt,arc=8pt,center title,
			title={\bfseries\color{white} \faWarehouse\ ALMACENAMIENTO}]
			\centering
			Costo de mantener una unidad en inventario durante un período específico.
		\end{tcolorbox}
		
		\begin{tcolorbox}[colback=naranjaclaro,colframe=naranjaacento,boxrule=2pt,arc=8pt,center title,
			title={\bfseries\color{white} \faCoins\ ADQUISICIÓN}]
			\centering
			Costo variable proporcional a la cantidad adquirida.
		\end{tcolorbox}
	\end{multicols}
	
	\subsection{Fórmula del EOQ}
	
	\begin{tcolorbox}[enhanced,colback=white,colframe=azulprincipal,boxrule=3pt,arc=10pt,
		drop shadow,title={\Large\bfseries\color{white} \faCalculator\ CANTIDAD ECONÓMICA DE PEDIDO}]
		
		\begin{equation}
			\boxed{\color{azulprincipal} Q^* = \sqrt{\frac{2DS}{H}}}
		\end{equation}
		
		\begin{center}
			\begin{tabular}{cl}
				\textcolor{azulprincipal}{\faArrowRight} & $Q^*$ = Cantidad óptima de pedido \\
				\textcolor{verdeprincipal}{\faArrowRight} & $D$ = Demanda anual \\
				\textcolor{rojoacento}{\faArrowRight} & $S$ = Costo de pedido por orden \\
				\textcolor{moradoacento}{\faArrowRight} & $H$ = Costo de almacenamiento por unidad por año \\
			\end{tabular}
		\end{center}
		
	\end{tcolorbox}
	
	\subsection{Ejercicio Resuelto: EOQ con Pedidos Retrasados}
	
	\begin{tcolorbox}[enhanced,colback=naranjaclaro,colframe=naranjaacento,boxrule=2pt,arc=8pt,
		drop shadow,title={\bfseries\color{white} \faPuzzlePiece\ PROBLEMA}]
		
		\textbf{Clínica de Optometría:} Una clínica vende 10,000 monturas anuales. El proveedor cobra \$15 por unidad, con costo de pedido de \$50. El costo de déficit es \$15 por montura/año por pérdida de negocios futuros. El costo de retención anual es 30\% del costo de compra.
		
	\end{tcolorbox}
	
	\begin{tcolorbox}[enhanced,colback=azulclaro,colframe=azulprincipal,boxrule=2pt,arc=8pt,
		title={\bfseries\color{white} \faCalculator\ SOLUCIÓN PASO A PASO}]
		
		\textbf{Paso 1: Identificación de parámetros}
		\begin{align}
			D &= 10,000 \text{ monturas/año}\\
			S &= \$50 \text{ por pedido}\\
			H &= 0.30 \times \$15 = \$4.5 \text{ por unidad/año}\\
			B &= \$15 \text{ por unidad/año (costo de déficit)}\\
			C &= \$15 \text{ por unidad}
		\end{align}
		
		\textbf{Paso 2: Cálculo de la cantidad óptima}
		
		\begin{equation}
			\boxed{Q^* = \sqrt{\frac{2DS}{H}} \times \sqrt{\frac{H + B}{B}}}
		\end{equation}
		
		\begin{align}
			Q^* &= \sqrt{\frac{2 \times 10,000 \times 50}{4.5}} \times \sqrt{\frac{4.5 + 15}{15}}\\
			&= \sqrt{222,222.22} \times \sqrt{1.3}\\
			&= 471.4 \times 1.14\\
			&= \textbf{537.48 $\approx$ 538 monturas}
		\end{align}
		
		\textbf{Paso 3: Cálculo del inventario máximo}
		
		\begin{align}
			I_{max} &= 537.48 \times \frac{15}{4.5 + 15} = \textbf{413.48 $\approx$ 414 monturas}
		\end{align}
		
		\textbf{Paso 4: Déficit máximo}
		
		\begin{align}
			\text{Déficit máximo} &= 537.48 - 413.48 = \textbf{124 monturas}
		\end{align}
		
	\end{tcolorbox}
	
	\begin{tcolorbox}[enhanced,colback=grisclaro,colframe=grisoScuro,boxrule=2pt,arc=8pt,
		title={\bfseries\color{white} \faTable\ TABLA DE RESULTADOS}]
		
		\begin{center}
			\small
			\begin{tabular}{|l|c|l|c|}
				\hline
				\rowcolor{azulclaro}
				\textbf{Parámetro} & \textbf{Valor} & \textbf{Parámetro} & \textbf{Valor} \\
				\hline
				Demand rate(D) & 10000 & Optimal order quantity (Q*) & \textcolor{verdeprincipal}{\textbf{537.48}} \\
				\hline
				Setup cost(S) & 50 & Maximum Inventory Level & \textcolor{azulprincipal}{\textbf{413.45}} \\
				\hline
				Holding cost(H) & 4.5 & Maximum Shortage & \textcolor{rojoacento}{\textbf{124.03}} \\
				\hline
				Backorder cost(B) & 15 & Orders per year & 18.61 \\
				\hline
				Unit cost & 15 & Total Cost & \textcolor{moradoacento}{\textbf{151860.5}} \\
				\hline
			\end{tabular}
		\end{center}
		
	\end{tcolorbox}
	
	\section{Modelo de Lote de Producción}
	
	\begin{tcolorbox}[enhanced,colback=verdeclaro,colframe=verdeprincipal,boxrule=3pt,arc=12pt,
		drop shadow,title={\Large\bfseries\color{white} \faCogs\ FUNDAMENTOS DEL MODELO}]
		
		El modelo de lote de producción se aplica cuando la empresa produce el artículo internamente en lugar de comprarlo a un proveedor externo. La principal diferencia con el EOQ básico es que el inventario se acumula gradualmente durante la producción.
		
	\end{tcolorbox}
	
	\subsection{Ejercicio Resuelto: Flemming Accessories}
	
	\begin{tcolorbox}[enhanced,colback=violetaclaro,colframe=moradoacento,boxrule=2pt,arc=8pt,
		drop shadow,title={\bfseries\color{white} \faPuzzlePiece\ PROBLEMA}]
		
		\textbf{Flemming Accessories:} Fabrica cortadoras de papel. Demanda anual: 6,750 unidades constantes. Kristen puede fabricar 125/día en promedio. Demanda durante producción: 30/día. Costo de preparación: \$150. Costo de almacenamiento: \$1/minicortadora/año.
		
	\end{tcolorbox}
	
	\begin{tcolorbox}[enhanced,colback=azulclaro,colframe=azulprincipal,boxrule=2pt,arc=8pt,
		title={\bfseries\color{white} \faCalculator\ SOLUCIÓN}]
		
		\textbf{Parámetros:}
		\begin{align}
			D &= 6,750 \text{ unidades/año}\\
			p &= 125 \times 225 = 28,125 \text{ unidades/año}\\
			Q^* &= \textbf{1,632 unidades}
		\end{align}
		
	\end{tcolorbox}
	
	\section{Modelo EOQ con Tiempo de Entrega}
	
	\begin{tcolorbox}[enhanced,colback=naranjaclaro,colframe=naranjaacento,boxrule=3pt,arc=12pt,
		drop shadow,title={\Large\bfseries\color{white} \faHourglass\ FUNDAMENTOS DEL MODELO}]
		
		Cuando existe un tiempo de entrega entre el momento del pedido y su recepción, es necesario determinar el punto de reorden para evitar faltantes.
		
	\end{tcolorbox}
	
	\subsection{Ejercicio Resuelto: ICR LLC}
	
	\begin{tcolorbox}[enhanced,colback=violetaclaro,colframe=moradoacento,boxrule=2pt,arc=8pt,
		drop shadow,title={\bfseries\color{white} \faPuzzlePiece\ PROBLEMA}]
		
		\textbf{ICR LLC:} El director de compras recibió una oferta de ANTSIS-SA para el producto A-2147, con precio unitario de \$22.00 y tiempo de entrega de 4 días. Este suministrador ofrece entregar el pedido de forma paulatina a una tasa de 2000 unidades al mes.
		
		\textbf{Datos:}
		\begin{itemize}[leftmargin=*,label=\textcolor{azulprincipal}{\faArrowRight}]
			\item D = 19500 componentes al año
			\item S = 24072 componentes al año
			\item K = \$22.00 por componente
			\item Co = \$50 por pedido
			\item Ch = (0.02)(12)(22.00) = \$5.28 por componente al año
			\item L = 4 días
			\item Días laborables en el año = 307
		\end{itemize}
		
	\end{tcolorbox}
	
	\begin{tcolorbox}[enhanced,colback=azulclaro,colframe=azulprincipal,boxrule=2pt,arc=8pt,
		title={\bfseries\color{white} \faCalculator\ SOLUCIÓN PASO A PASO}]
		
		\textbf{Paso 1: Cálculo de cantidad óptima}
		\begin{equation}
			Q = \sqrt{\frac{2(19500)50}{5.28(1-19500/24072)}} = 1394.19 \approx 1395 \text{ componentes}
		\end{equation}
		
		\textbf{Paso 2: Punto de reorden}
		\begin{align}
			R &= \frac{19500 \times 4}{307} = 254 \text{ componentes}
		\end{align}
		
		\textbf{Paso 3: Cálculos adicionales}
		\begin{itemize}[leftmargin=*,label=\textcolor{verdeprincipal}{\faCheck}]
			\item Tiempo entre pedidos: $t = 1395/19500 = 0.071 \approx 21.79$ días
			\item Número de pedidos: $n = 19500/1395 = 13.9$ pedidos
			\item Período de reabastecimiento: $L = 1395/24072 = 0.057 \approx 17.49$ días
			\item Inventario máximo: $I_{max} = (24072 - 19500) \times 0.057 \approx 261$ unidades
		\end{itemize}
		
	\end{tcolorbox}
	
	\section{Modelo con Descuentos por Cantidad}
	
	\begin{tcolorbox}[enhanced,colback=violetaclaro,colframe=moradoacento,boxrule=3pt,arc=12pt,
		drop shadow,title={\Large\bfseries\color{white} \faTag\ FUNDAMENTOS DEL MODELO}]
		
		Los descuentos por cantidad son estrategias de precios donde el costo unitario disminuye según la cantidad comprada.
		
	\end{tcolorbox}
	
	\subsection{Ejercicio Resuelto: MBI Computadoras}
	
	\begin{tcolorbox}[enhanced,colback=naranjaclaro,colframe=naranjaacento,boxrule=2pt,arc=8pt,
		drop shadow,title={\bfseries\color{white} \faPuzzlePiece\ PROBLEMA}]
		
		\textbf{MBI:} Fabrica computadoras personales. Todas sus computadoras usan un disco duro que compra a Ynos. La fábrica opera 52 semanas por año y debe ensamblar 100 discos duros en las computadoras por semana. La tasa de costo de mantener es igual a 20\% del valor del inventario. El costo administrativo de colocar órdenes con Ynos se estima en 50 dólares.
		
	\end{tcolorbox}
	
	\begin{tcolorbox}[enhanced,colback=grisclaro,colframe=grisoScuro,boxrule=2pt,arc=8pt,
		title={\bfseries\color{white} \faTable\ ESTRUCTURA DE DESCUENTOS}]
		
		\begin{center}
			\begin{tabular}{|c|c|c|}
				\hline
				\rowcolor{azulclaro}
				\textbf{Categoría} & \textbf{Cantidad} & \textbf{Precio} \\
				\hline
				1 & 1 a 99 & \textcolor{rojoacento}{\$100} \\
				\hline
				2 & 100 a 499 & \textcolor{naranjaacento}{\$95} \\
				\hline
				3 & 500 o más & \textcolor{verdeprincipal}{\$90} \\
				\hline
			\end{tabular}
		\end{center}
		
		\textbf{Parámetros:} D = 5,200 discos/año, S = \$50 por pedido, i = 20\%
		
		\textbf{Análisis por categorías:}
		\begin{itemize}[leftmargin=*,label=\textcolor{azulprincipal}{\faArrowRight}]
			\item \textbf{Q = 99:} Costo total = \textcolor{rojoacento}{\$523,616.3}
			\item \textbf{Q = 165:} Costo total = \textcolor{naranjaacento}{\$497,143.3}
			\item \textbf{Q = 500:} Costo total = \textcolor{verdeprincipal}{\$473,020.0} \faCheckCircle\ ÓPTIMO
		\end{itemize}
		
		\textbf{Decisión:} Q* óptimo = 500 unidades, Costo total anual = \$473,020.0
		
	\end{tcolorbox}
	
	\section{Modelo de Demanda Discreta Sin Inventario Inicial}
	
	\begin{tcolorbox}[enhanced,colback=naranjaclaro,colframe=naranjaacento,boxrule=3pt,arc=12pt,
		drop shadow,title={\Large\bfseries\color{white} \faChartLine\ MODELO DE DEMANDA DISCRETA}]
		
		Cuando la demanda sigue una distribución de probabilidad discreta, utilizamos el modelo de cantidad crítica.
		
	\end{tcolorbox}
	
	\subsection{Ejercicio Resuelto: Producto con Demanda Incierta}
	
	\begin{tcolorbox}[enhanced,colback=violetaclaro,colframe=moradoacento,boxrule=2pt,arc=8pt,
		drop shadow,title={\bfseries\color{white} \faPuzzlePiece\ PROBLEMA}]
		
		El costo de compra por unidad de un producto es \$10 y su costo de tenerlo en inventario por unidad por período es de \$1. La pérdida por demandas postergadas ocasiona un costo de \$15 por unidad. ¿Cuál es la cantidad óptima a ordenar dada la distribución de probabilidad de la demanda?
		
	\end{tcolorbox}
	
	\begin{tcolorbox}[enhanced,colback=azulclaro,colframe=azulprincipal,boxrule=2pt,arc=8pt,
		title={\bfseries\color{white} \faCalculator\ SOLUCIÓN}]
		
		\textbf{Parámetros:}
		\begin{itemize}[leftmargin=*,label=\textcolor{verdeprincipal}{\faArrowRight}]
			\item Costo de compra: $c = \$10$
			\item Costo de almacenamiento: $h = \$1$
			\item Costo de escasez: $p = \$15$
		\end{itemize}
		
		\textbf{Proporción crítica:}
		\begin{equation}
			\boxed{\frac{p - c}{p + h} = \frac{15 - 10}{15 + 1} = 0.3125}
		\end{equation}
		
		\begin{center}
			\small
			\begin{tabular}{|c|c|c|}
				\hline
				\rowcolor{azulclaro}
				\textbf{d} & \textbf{P$_D$(d)} & \textbf{$\sum$P$_D$(d)} \\
				\hline
				0 & 0.05 & 0.05 \\
				1 & 0.10 & 0.15 \\
				2 & 0.10 & 0.25 \\
				3 & 0.20 & 0.45 \\
				4 & 0.25 & 0.70 \\
				5 & 0.15 & 0.85 \\
				6 & 0.05 & 0.90 \\
				7 & 0.05 & 0.95 \\
				8 & 0.05 & 1.00 \\
				\hline
			\end{tabular}
		\end{center}
		
		Como P(D $\leq$ 2) = 0.25 $<$ 0.3125 y P(D $\leq$ 3) = 0.45 $>$ 0.3125
		
		\textbf{Cantidad óptima:} $y^* = 3$ unidades
		
	\end{tcolorbox}
	
	\section{Modelo Probabilístico de Un Período}
	
	\subsection{Ejercicio Resuelto: Chicago Cheese}
	
	\begin{tcolorbox}[enhanced,colback=verdeclaro,colframe=verdeprincipal,boxrule=2pt,arc=8pt,
		drop shadow,title={\bfseries\color{white} \faPuzzlePiece\ PROBLEMA}]
		
		Teresa Granger es gerente de Chicago Cheese, que elabora quesos para untar y otros productos de queso relacionados. E-Z Spread Cheese es un producto que siempre ha sido muy popular. 
		
		\textbf{Datos del problema:}
		\begin{itemize}[leftmargin=*,label=\textcolor{azulprincipal}{\faArrowRight}]
			\item Precio de venta: \$100 por caja
			\item Costo de producción: \$75 por caja
			\item Valor de salvamento: \$50 por caja (productos no vendidos)
			\item Teresa nunca vende queso de más de una semana
		\end{itemize}
		
	\end{tcolorbox}
	
	\begin{tcolorbox}[enhanced,colback=azulclaro,colframe=azulprincipal,boxrule=2pt,arc=8pt,
		title={\bfseries\color{white} \faCalculator\ SOLUCIÓN}]
		
		\textbf{Costos marginales:}
		\begin{align}
			Co &= 100 - 75 = \$25 \text{ (costo de oportunidad por falta)}\\
			Cu &= 75 - 50 = \$25 \text{ (costo de exceso)}
		\end{align}
		
		\textbf{Proporción crítica:}
		\begin{equation}
			\frac{Cu}{Co + Cu} = \frac{25}{25 + 25} = 0.50
		\end{equation}
		
		\begin{center}
			\small
			\begin{tabular}{|c|c|c|c|c|c|}
				\hline
				\rowcolor{azulclaro}
				\textbf{Demanda} & \textbf{10} & \textbf{11} & \textbf{12} & \textbf{13} & \textbf{14} \\
				\hline
				Probabilidad & 0.2 & 0.3 & 0.2 & 0.2 & 0.1 \\
				\hline
				Prob. Acum. & 0.2 & 0.5 & 0.7 & 0.9 & 1.0 \\
				\hline
			\end{tabular}
		\end{center}
		
		Como la proporción crítica es 0.50, y P(D $\leq$ 11) = 0.5
		
		\textbf{Respuesta:} Producir \textbf{11 o 12 cajas de queso}
		
	\end{tcolorbox}
	
	\section{Modelo de Tienda de Abarrotes con Demanda Probabilística}
	
	\subsection{Ejercicio Resuelto: Gestión con Demanda Normal}
	
	\begin{tcolorbox}[enhanced,colback=violetaclaro,colframe=moradoacento,boxrule=2pt,arc=8pt,
		drop shadow,title={\bfseries\color{white} \faPuzzlePiece\ PROBLEMA}]
		
		Una tienda de abarrotes atiende una demanda anual de un cierto producto, la cual se distribuye como una Normal con promedio 1000 cajas con desviación estándar = 40.8 cajas. El costo de hacer un pedido es \$50 y demora en llegar 2 semanas. El costo anual por conservar una caja en inventario es \$10. A la larga, debe cumplirse la demanda (no hay ventas perdidas). El costo del agotamiento de las existencias por caja es de \$20. Considere 52 semanas/año.
		
	\end{tcolorbox}
	
	\begin{tcolorbox}[enhanced,colback=azulclaro,colframe=azulprincipal,boxrule=2pt,arc=8pt,
		title={\bfseries\color{white} \faCalculator\ SOLUCIÓN}]
		
		\textbf{Parámetros del problema:}
		\begin{align}
			\mu_0 &= 1000 \text{ cajas/año}\\
			\sigma_0 &= 40.8 \text{ cajas/año}\\
			k &= \$50 \text{ por pedido}\\
			L &= 2 \text{ semanas}\\
			h &= \$10 \text{ por caja/año}\\
			C_u &= \$20 \text{ por caja}
		\end{align}
		
		\textbf{Demanda durante el lead time:}
		\begin{align}
			\mu_L &= \frac{1000 \times 2}{52} = 38.5 \text{ cajas}\\
			\sigma_L &= \frac{40.8 \times 2}{52} = 8.002 \text{ cajas}
		\end{align}
		
		\textbf{Resultados óptimos:}
		\begin{itemize}[leftmargin=*,label=\textcolor{verdeprincipal}{\faCheckCircle}]
			\item $Q^* = 100.00$ cajas
			\item $R^* = 51.62$ cajas (punto de reorden)
			\item Número de pedidos por año: 10.00
		\end{itemize}
		
	\end{tcolorbox}
	
	\section{Modelo de Producción con Pedidos Atrasados}
	
	\subsection{Ejercicio Resuelto: Empresa Ladrillera}
	
	\begin{tcolorbox}[enhanced,colback=naranjaclaro,colframe=naranjaacento,boxrule=2pt,arc=8pt,
		drop shadow,title={\bfseries\color{white} \faPuzzlePiece\ PROBLEMA}]
		
		Una empresa ladrillera tiene una demanda anual de 210,000 ladrillos. Dicha empresa los produce a un ritmo mensual de 37,500 ladrillos. Se incurre en un costo de \$450 cada vez que se realiza una corrida de producción, el costo anual de almacenamiento es de 1.2 \$/unidad, y el costo anual por tener demanda pendiente es 0.5 \$/unidad. Asumir 360 días por año.
		
		\textbf{Se pide:} 1) El tamaño óptimo de cada corrida de producción, 2) La cantidad de corridas de producción que se deben hacer cada año, 3) El costo total, 4) El nivel máximo de inventario, 5) La demanda pendiente máxima.
		
	\end{tcolorbox}
	
	\begin{tcolorbox}[enhanced,colback=azulclaro,colframe=azulprincipal,boxrule=2pt,arc=8pt,
		title={\bfseries\color{white} \faCalculator\ RESULTADOS}]
		
		\textbf{Parámetros calculados:}
		\begin{itemize}[leftmargin=*,label=\textcolor{azulprincipal}{\faArrowRight}]
			\item Demanda diaria: 210,000/360 = 583.33 ladrillos/día
			\item Producción diaria: 37,500 × 12/360 = 1,250 ladrillos/día
		\end{itemize}
		
		\textbf{Resultados del modelo:}
		\begin{center}
			\small
			\begin{tabular}{|l|c|}
				\hline
				\rowcolor{azulclaro}
				\textbf{Parámetro} & \textbf{Valor} \\
				\hline
				Cantidad óptima (Q*) & 31,686.95 ladrillos \\
				\hline
				Inventario máximo & 10,537.46 ladrillos \\
				\hline
				Déficit máximo & 11,929.2 ladrillos \\
				\hline
				Corridas por año & 6.63 \\
				\hline
				Costo total anual & \$5,964.6 \\
				\hline
			\end{tabular}
		\end{center}
		
	\end{tcolorbox}
	
	\section{Modelo EOQ con Desabastecimientos}
	
	\subsection{Ejercicio Resuelto: Sistema con Escasez Planificada}
	
	\begin{tcolorbox}[enhanced,colback=verdeclaro,colframe=verdeprincipal,boxrule=2pt,arc=8pt,
		drop shadow,title={\bfseries\color{white} \faPuzzlePiece\ PROBLEMA}]
		
		Un administrador de un sistema de inventario ha utilizado durante años el modelo del lote económico en su trabajo, pero piensa que incorporando escasez planificada debe obtener un resultado más rentable. ¿Es esto correcto?
		
		Considerar un producto cuya demanda es de 800 unidades, cuyo costo de mantener en inventario una unidad durante un mes es de 0.25 € cuyo costo de ordenamiento es de 150 € y cuyo costo de no poseer una unidad durante un año es de 20 €.
		
		\textbf{Preguntas:}
		a) ¿Cuál es el ahorro que se produce si utilizamos el modelo de desabastecimientos permitidos?
		b) Si el tiempo de entrega es de 1 mes, ¿cuál es el punto de reorden?
		
	\end{tcolorbox}
	
	\begin{tcolorbox}[enhanced,colback=azulclaro,colframe=azulprincipal,boxrule=2pt,arc=8pt,
		title={\bfseries\color{white} \faCalculator\ SOLUCIÓN}]
		
		\textbf{Parámetros del modelo:}
		\begin{align}
			D &= 800 \text{ unidades/año}\\
			C_o &= 150 \text{ € por pedido}\\
			C_s &= 20 \text{ € por unidad/año (escasez)}\\
			C_h &= 0.25 \times 12 = 3 \text{ € por unidad/año}
		\end{align}
		
		\textbf{Modelo EOQ básico:}
		\begin{equation}
			Q_{EOQ} = \sqrt{\frac{2 \times 150 \times 800}{3}} = 282.843 \text{ unidades}
		\end{equation}
		
		\textbf{Modelo con desabastecimientos:}
		\begin{align}
			Q^* &= \sqrt{\frac{2 \times 150 \times 800}{3}} \times \sqrt{\frac{3 + 20}{20}} = 303.315 \text{ unidades}\\
			S^* &= \frac{20}{3 + 20} \times 303.315 = 263.752 \text{ unidades}
		\end{align}
		
		\textbf{Resultados:}
		\begin{itemize}[leftmargin=*,label=\textcolor{verdeprincipal}{\faCheckCircle}]
			\item Costo EOQ tradicional: 848.528 €
			\item Costo con desabastecimientos: 791.256 €
			\item \textbf{Ahorro: 57.272 €} (justifica la escasez planificada)
			\item Punto de reorden (L = 1 mes): R = 66.666 unidades
		\end{itemize}
		
	\end{tcolorbox}
	
	\section{Modelo Probabilístico de Revisión Periódica (R,S)}
	
	\subsection{Ejercicio Resuelto: Modelo (R,S)}
	
	\begin{tcolorbox}[enhanced,colback=violetaclaro,colframe=moradoacento,boxrule=2pt,arc=8pt,
		drop shadow,title={\bfseries\color{white} \faPuzzlePiece\ PROBLEMA}]
		
		Sea un producto cuyo costo es de c = \$60.00, pero si no se dispone de él cuando se necesita, ocasiona una pérdida de \$800.00 y si se ha comprado y no se utiliza debe pagarse un costo de almacenamiento de \$10.00 por período R. La demanda es discreta con probabilidades para cada cantidad dadas por una tabla específica.
		
	\end{tcolorbox}
	
	\begin{tcolorbox}[enhanced,colback=azulclaro,colframe=azulprincipal,boxrule=2pt,arc=8pt,
		title={\bfseries\color{white} \faCalculator\ SOLUCIÓN}]
		
		\textbf{Parámetros:}
		\begin{itemize}[leftmargin=*,label=\textcolor{azulprincipal}{\faArrowRight}]
			\item Costo del producto: c = \$60.00
			\item Costo de ruptura: p = \$800.00
			\item Costo de almacenamiento: h = \$10.00 por período R
		\end{itemize}
		
		\textbf{Proporción crítica:}
		\begin{equation}
			\frac{p}{h + p} = \frac{800}{10 + 800} = 0.98765432
		\end{equation}
		
		\begin{center}
			\small
			\begin{tabular}{|c|c|c|c|c|c|c|}
				\hline
				\rowcolor{azulclaro}
				\textbf{d} & \textbf{0} & \textbf{1} & \textbf{2} & \textbf{3} & \textbf{4} & \textbf{5+} \\
				\hline
				P(d) & 0.1 & 0.2 & 0.2 & 0.3 & 0.1 & 0.1 \\
				\hline
				P(D $\leq$ d) & 0.1 & 0.3 & 0.5 & 0.8 & 0.9 & 1.0 \\
				\hline
			\end{tabular}
		\end{center}
		
		El modelo utiliza la expresión $M(D \leq q^*) = p/(h + p) = 0.98765432$
		
		Como P(D $\leq$ 4) = 0.9 $<$ 0.98765432 y P(D $\leq$ 5) = 1.0 $>$ 0.98765432
		
		\textbf{Decisión óptima:} Ordenar hasta el nivel S = \textbf{4 unidades}
		
	\end{tcolorbox}
	
	\section{Conclusiones}
	
	\begin{tcolorbox}[enhanced,colback=grisclaro,colframe=grisoScuro,boxrule=3pt,arc=15pt,
		drop shadow,title={\Large\bfseries\color{white} \faFlag\ SÍNTESIS DE APRENDIZAJES}]
		
		\textbf{Flexibilidad de los Modelos:} Los modelos de inventarios ofrecen herramientas adaptables desde EOQ básico hasta modelos probabilísticos complejos.
		
		\textbf{Importancia del Análisis de Costos:} El factor crítico es el correcto análisis de costos: ordenamiento, almacenamiento, escasez y adquisición.
		
		\textbf{Impacto de la Incertidumbre:} Los modelos probabilísticos permiten gestión realista incorporando distribuciones de probabilidad.
		
		\textbf{Resumen de Ejercicios Resueltos:}
		\begin{enumerate}[leftmargin=*,label=\textcolor{azulprincipal}{\arabic*.}]
			\item \textbf{EOQ con pedidos retrasados:} Clínica de optometría - Q* = 538 monturas
			\item \textbf{Lote de producción:} Flemming Accessories - Q* = 1,632 minicortadoras  
			\item \textbf{EOQ con tiempo de entrega:} ICR LLC - Q* = 1,395 componentes, R = 254
			\item \textbf{Descuentos por cantidad:} MBI Computadoras - Q* = 500 discos, Ahorro significativo
			\item \textbf{Demanda discreta:} Producto incierto - Q* = 3 unidades
			\item \textbf{Modelo probabilístico:} Chicago Cheese - Producir 11 o 12 cajas
			\item \textbf{Demanda normal:} Tienda de abarrotes - Q* = 100 cajas, R* = 52 cajas
			\item \textbf{Producción con atrasos:} Empresa ladrillera - Q* = 31,687 ladrillos
			\item \textbf{EOQ con desabastecimientos:} Ahorro de 57.272 € vs EOQ tradicional
			\item \textbf{Modelo (R,S):} Sistema de revisión periódica - Nivel S = 4 unidades
		\end{enumerate}
		
	\end{tcolorbox}
	
	\section{Recomendaciones}
	
	\begin{multicols}{2}
		\begin{tcolorbox}[colback=verdeclaro,colframe=verdeprincipal,boxrule=2pt,arc=8pt,
			title={\bfseries\color{white} \faLaptop\ TECNOLOGÍA}]
			\begin{itemize}[leftmargin=*,label=\textcolor{azulprincipal}{\faCheck}]
				\item Sistemas ERP integrados
				\item Análisis de datos avanzado  
				\item Internet de las cosas (IoT)
				\item Inteligencia artificial
			\end{itemize}
		\end{tcolorbox}
		
		\begin{tcolorbox}[colback=azulclaro,colframe=azulprincipal,boxrule=2pt,arc=8pt,
			title={\bfseries\color{white} \faUsers\ DESARROLLO}]
			\begin{itemize}[leftmargin=*,label=\textcolor{verdeprincipal}{\faCheck}]
				\item Capacitación del personal
				\item Cultura de mejora continua
				\item Colaboración interdisciplinaria
				\item Benchmarking industrial
			\end{itemize}
		\end{tcolorbox}
	\end{multicols}
	
	\begin{tcolorbox}[enhanced,colback=naranjaclaro,colframe=naranjaacento,boxrule=3pt,arc=12pt,
		drop shadow,title={\Large\bfseries\color{white} \faGlobe\ TENDENCIAS FUTURAS}]
		
		\textbf{Evolución hacia:}
		\begin{itemize}[leftmargin=*,label=\textcolor{azulprincipal}{\faArrowRight}]
			\item Mayor sofisticación matemática con optimización avanzada
			\item Integración multiobjetivo considerando múltiples criterios
			\item Adaptabilidad dinámica con ajustes automáticos
			\item Sostenibilidad incorporando criterios ambientales
		\end{itemize}
		
	\end{tcolorbox}
	
	\begin{center}
		\begin{tcolorbox}[enhanced,colback=moradoacento,colframe=white,boxrule=2pt,arc=10pt,
			width=0.8\textwidth,center]
			\centering
			{\Large\bfseries\color{white} \faTrophy\ La gestión óptima de inventarios seguirá siendo un factor crítico de éxito empresarial}
		\end{tcolorbox}
	\end{center}
	
\end{document}
