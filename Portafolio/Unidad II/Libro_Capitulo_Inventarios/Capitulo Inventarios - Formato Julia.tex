\documentclass[12pt,a4paper]{book}
\usepackage[utf8]{inputenc}
\usepackage[spanish]{babel}
\usepackage{amsmath}
\usepackage{amsfonts}
\usepackage{amssymb}
\usepackage{graphicx}
\usepackage{geometry}
\usepackage{fancyhdr}
\usepackage{array}
\usepackage{booktabs}
\usepackage{float}
\usepackage{xcolor}
\usepackage{tikz}
\usepackage{pgfplots}
\usepackage[most]{tcolorbox}
\usepackage{fontawesome5}
\usepackage{enumitem}
\usepackage{mdframed}
\usepackage{listings}
\usepackage{multicol}
\usepackage{colortbl}

\geometry{left=2.5cm,right=2.5cm,top=3cm,bottom=3cm}
\setlength{\headheight}{30pt}
\pgfplotsset{compat=1.18}

% Definir paleta de colores moderna
\definecolor{azulprincipal}{RGB}{41, 98, 255}
\definecolor{azulsecundario}{RGB}{0, 123, 255}
\definecolor{verdeprincipal}{RGB}{40, 167, 69}
\definecolor{verdesecundario}{RGB}{25, 135, 84}
\definecolor{naranjaacento}{RGB}{255, 193, 7}
\definecolor{rojoacento}{RGB}{220, 53, 69}
\definecolor{moradoacento}{RGB}{111, 66, 193}
\definecolor{grisclaro}{RGB}{255, 248, 220}
\definecolor{grisoScuro}{RGB}{108, 117, 125}
\definecolor{azulclaro}{RGB}{179, 229, 252}
\definecolor{verdeclaro}{RGB}{212, 237, 218}
\definecolor{naranjaclaro}{RGB}{255, 243, 205}
\definecolor{violetaclaro}{RGB}{232, 222, 248}
\definecolor{crema}{RGB}{255, 248, 220}
\definecolor{juliacolor}{RGB}{149, 88, 178}
\definecolor{juliablue}{RGB}{56, 152, 222}

% Configuraciones para cajas coloridas
\tcbuselibrary{skins,breakable}

% Configuración adicional para evitar problemas de cajas
\setlength{\parskip}{6pt}
\tolerance=1000
\hbadness=10000
\vbadness=10000

% Configuración de encabezados
\pagestyle{fancy}
\fancyhf{}
\fancyhead[LE,RO]{\colorbox{azulprincipal}{\textcolor{white}{\bfseries\thepage}}}
\fancyhead[LO,RE]{\textcolor{azulprincipal}{\bfseries Cap\'itulo: Inventarios con Julia}}
\renewcommand{\headrulewidth}{2pt}
\renewcommand{\headrule}{\hbox to\headwidth{\color{azulprincipal}\leaders\hrule height \headrulewidth\hfill}}

\lstdefinelanguage{Julia}{
	keywords={function, end, if, else, elseif, for, while, do, try, catch, finally, return, break, continue, macro, quote, let, local, global, const, struct, mutable, abstract, primitive, type, where, import, using, export, module, baremodule},
	keywordstyle=\color{juliablue}\bfseries,
	ndkeywords={true, false, nothing, NaN, Inf},
	ndkeywordstyle=\color{rojoacento}\bfseries,
	identifierstyle=\color{black},
	sensitive=true,
	comment=[l]{\#},
	commentstyle=\color{grisoScuro}\ttfamily,
	stringstyle=\color{verdeprincipal}\ttfamily,
	numbers=left,
	numberstyle=\tiny\color{grisoScuro},
	numbersep=10pt,
	tabsize=2,
	showtabs=false,
	showspaces=false,
	showstringspaces=false,
	extendedchars=false,
	breaklines=true,
	postbreak=\mbox{\textcolor{red}{$\hookrightarrow$}\space},
	frame=tb,
	framerule=1pt,
	rulecolor=\color{juliacolor},
	backgroundcolor=\color{grisclaro},
	inputencoding=utf8
}

\lstset{
	language=Julia,
	basicstyle=\footnotesize\ttfamily,
	keywordstyle=\color{juliablue}\bfseries,
	commentstyle=\color{grisoScuro}\ttfamily,
	stringstyle=\color{verdeprincipal}\ttfamily,
	breaklines=true,
	breakatwhitespace=true,
	columns=flexible
}

\usetikzlibrary{shapes.geometric, arrows, shadows, patterns}

\begin{document}
	
	% PORTADA
	\begin{titlepage}
		\begin{tikzpicture}[remember picture,overlay]
			\fill[crema] (current page.south west) rectangle (current page.north east);
			\fill[azulsecundario,opacity=0.7] (current page.south west) rectangle ([yshift=8cm]current page.south east);
			\fill[verdeprincipal,opacity=0.5] (current page.south west) rectangle ([yshift=4cm]current page.south east);
			
			\foreach \i in {1,...,20}
			{
				\fill[white,opacity=0.1] ([xshift=\i cm, yshift=\i cm]current page.south west) circle (0.5);
			}
			
			\draw[white,line width=3pt,opacity=0.8] ([yshift=2cm]current page.south west) -- ([yshift=2cm]current page.south east);
			\draw[naranjaacento,line width=2pt] ([yshift=2.2cm]current page.south west) -- ([yshift=2.2cm]current page.south east);
		\end{tikzpicture}
		
		\centering
		\vspace{1.5cm}
		
		\begin{tcolorbox}[colback=white,colframe=azulprincipal,boxrule=3pt,arc=15pt,drop shadow]
			\centering
			{\Large\bfseries\color{azulprincipal} \faUniversity\ UNIVERSIDAD NACIONAL DEL ALTIPLANO}\\[0.3cm]
			{\large\color{grisoScuro} Ingenier\'ia Estad\'istica e Inform\'atica}
		\end{tcolorbox}
		
		\vspace{1.5cm}
		
		\begin{tcolorbox}[colback=naranjaclaro,colframe=naranjaacento,boxrule=3pt,arc=15pt,drop shadow]
			\centering
			{\Huge\bfseries\color{rojoacento} INVENTARIOS CON JULIA}\\[0.5cm]
			{\Large\color{grisoScuro} \faBoxes\ Modelos de Gesti\'on \'Optima Computacional}
		\end{tcolorbox}
		
		\vspace{1.5cm}
		
		\begin{tikzpicture}
			\node[azulprincipal,scale=3] at (-3,0) {\faCalculator};
			\node[juliacolor,scale=3] at (0,0) {\faCode};
			\node[rojoacento,scale=3] at (3,0) {\faCogs};
		\end{tikzpicture}
		
		\vspace{1.5cm}
		
		\begin{tcolorbox}[colback=violetaclaro,colframe=moradoacento,boxrule=2pt,arc=10pt]
			\centering
			{\large\bfseries\color{moradoacento} \faLightbulb\ EOQ • Descuentos • Probabil\'isticos • Optimizaci\'on}\\[0.3cm]
			{\normalsize\color{grisoScuro} Implementaci\'on Computacional con Lenguaje Julia}
		\end{tcolorbox}
		
		\vfill
		
		\begin{tcolorbox}[colback=grisclaro,colframe=grisoScuro,boxrule=2pt,arc=8pt,drop shadow]
			\centering
			{\Large\bfseries\color{azulprincipal} \faUser\ AUTOR}\\[0.5cm]
			{\LARGE\bfseries\color{rojoacento} Etzel Yuliza Peralta L\'opez}\\[0.3cm]
			{\large\color{grisoScuro} Material de Estudio Computacional con Julia}
		\end{tcolorbox}
		
	\end{titlepage}
	
	\newpage
	
	% ÍNDICE
	\begin{tcolorbox}[colback=azulclaro,colframe=azulprincipal,boxrule=2pt,arc=10pt,title={\Large\bfseries\color{white} \faList\ CONTENIDO DEL CAP\'ITULO}]
		\tableofcontents
	\end{tcolorbox}
	
	\newpage
	
	\chapter{Inventarios con Julia}
	
	\begin{tcolorbox}[colback=naranjaclaro,colframe=naranjaacento,boxrule=2pt,arc=10pt,title={\large\bfseries\color{white} \faInfoCircle\ INTRODUCCI\'ON}]
		
		Los modelos de inventarios constituyen una herramienta fundamental en la gesti\'on empresarial moderna. En este cap\'itulo implementaremos estos modelos utilizando el lenguaje de programaci\'on Julia, que ofrece excelente rendimiento para c\'alculos matem\'aticos y optimizaci\'on.
		
		Julia combina la facilidad de uso de Python con la velocidad de C++, siendo ideal para resolver problemas complejos de optimizaci\'on en inventarios.
		
	\end{tcolorbox}
	
	\section{Configuraci\'on del Entorno Julia}
	
	\begin{tcolorbox}[enhanced,colback=grisclaro,colframe=juliacolor,boxrule=2pt,arc=8pt,
		drop shadow,title={\bfseries\color{white} \faCode\ PAQUETES NECESARIOS}]
		
		\begin{lstlisting}[language=Julia,basicstyle=\footnotesize\ttfamily]
			# Instalacion de paquetes necesarios
			using Pkg
			Pkg.add(["Optim", "Distributions", "Plots", 
			"DataFrames", "Printf"])
			
			# Importar librerias
			using Optim
			using Distributions
			using Plots
			using DataFrames
			using Printf
			using Statistics
		\end{lstlisting}
		
	\end{tcolorbox}
	
	\section{Modelo EOQ B\'asico}
	
	\begin{tcolorbox}[enhanced,colback=azulclaro,colframe=azulprincipal,boxrule=3pt,arc=12pt,
		drop shadow,title={\Large\bfseries\color{white} \faLightbulb\ IMPLEMENTACI\'ON EOQ}]
		
		\begin{lstlisting}[language=Julia,basicstyle=\footnotesize\ttfamily]
			"""
			Modelo EOQ (Economic Order Quantity) basico
			Parametros:
			- D: Demanda anual
			- S: Costo de pedido por orden
			- H: Costo de almacenamiento por unidad por año
			"""
			function eoq_basic(D, S, H)
			Q_star = sqrt(2 * D * S / H)
			total_cost = sqrt(2 * D * S * H)
			orders_per_year = D / Q_star
			time_between_orders = Q_star / D * 365  # dias
			
			return (
			Q_star = Q_star,
			total_cost = total_cost,
			orders_per_year = orders_per_year,
			time_between_orders = time_between_orders
			)
			end
			
			# Funcion para mostrar resultados
			function print_eoq_results(result, title="Resultados EOQ")
			println("="^50)
			println(title)
			println("="^50)
			@printf("Cantidad optima (Q*): %.2f unidades\n", 
			result.Q_star)
			@printf("Costo total anual: \$%.2f\n", 
			result.total_cost)
			@printf("Pedidos por año: %.2f\n", 
			result.orders_per_year)
			@printf("Tiempo entre pedidos: %.2f dias\n", 
			result.time_between_orders)
			println("="^50)
			end
		\end{lstlisting}
		
	\end{tcolorbox}
	
	\subsection{Ejercicio Resuelto: EOQ con Pedidos Retrasados}
	
	\begin{tcolorbox}[enhanced,colback=naranjaclaro,colframe=naranjaacento,boxrule=2pt,arc=8pt,
		drop shadow,title={\bfseries\color{white} \faPuzzlePiece\ PROBLEMA}]
		
		\textbf{Cl\'inica de Optometr\'ia:} Una cl\'inica vende 10,000 monturas anuales. El proveedor cobra \$15 por unidad, con costo de pedido de \$50. El costo de d\'eficit es \$15 por montura/a\~no. El costo de retenci\'on anual es 30\% del costo de compra.
		
	\end{tcolorbox}
	
	\begin{tcolorbox}[enhanced,colback=azulclaro,colframe=azulprincipal,boxrule=2pt,arc=8pt,
		title={\bfseries\color{white} \faCode\ SOLUCI\'ON EN JULIA},breakable]
		
		\begin{lstlisting}[language=Julia,basicstyle=\footnotesize\ttfamily]
			"""
			EOQ con pedidos retrasados (backorders permitidos)
			Parametros adicionales:
			- B: Costo de deficit por unidad por año
			"""
			function eoq_backorders(D, S, H, B)
			# Cantidad optima con backorders
			Q_star = sqrt(2 * D * S / H) * sqrt((H + B) / B)
			
			# Inventario maximo
			I_max = Q_star * (B / (H + B))
			
			# Deficit maximo
			deficit_max = Q_star - I_max
			
			# Costo total
			total_cost = sqrt(2 * D * S * H * B / (H + B))
			
			# Pedidos por año
			orders_per_year = D / Q_star
			
			return (
			Q_star = Q_star,
			I_max = I_max,
			deficit_max = deficit_max,
			total_cost = total_cost,
			orders_per_year = orders_per_year
			)
			end
			
			# Datos del problema de la clinica
			D = 10000  # monturas/año
			S = 50     # $ por pedido
			unit_cost = 15  # $ por unidad
			H = 0.30 * unit_cost  # $ por unidad/año
			B = 15     # $ por unidad/año (costo de deficit)
			
			# Resolver el problema
			resultado_clinica = eoq_backorders(D, S, H, B)
			
			println("CLINICA DE OPTOMETRIA - EOQ CON BACKORDERS")
			println("="^60)
			@printf("Cantidad optima (Q*): %.2f monturas\n", 
			resultado_clinica.Q_star)
			@printf("Inventario maximo: %.2f monturas\n", 
			resultado_clinica.I_max)
			@printf("Deficit maximo: %.2f monturas\n", 
			resultado_clinica.deficit_max)
			@printf("Costo total anual: \$%.2f\n", 
			resultado_clinica.total_cost)
			@printf("Pedidos por año: %.2f\n", 
			resultado_clinica.orders_per_year)
		\end{lstlisting}
		
	\end{tcolorbox}
	
	\section{Modelo de Lote de Producci\'on}
	
	\begin{tcolorbox}[enhanced,colback=verdeclaro,colframe=verdeprincipal,boxrule=3pt,arc=12pt,
		drop shadow,title={\Large\bfseries\color{white} \faCogs\ MODELO DE PRODUCCI\'ON},breakable]
		
		\begin{lstlisting}[language=Julia,basicstyle=\scriptsize\ttfamily]
			"""
			Modelo de lote de produccion (Production Lot Size)
			Parametros:
			- D: Demanda anual
			- S: Costo de preparacion
			- H: Costo de almacenamiento por unidad por año
			- P: Tasa de produccion anual
			"""
			function production_lot_model(D, S, H, P)
			if P <= D
			error("La tasa de produccion debe ser mayor que la demanda")
			end
			
			# Cantidad optima de produccion
			Q_star = sqrt(2 * D * S / H) * sqrt(P / (P - D))
			
			# Inventario maximo
			I_max = Q_star * (1 - D/P)
			
			# Tiempo de produccion
			t_production = Q_star / P * 365  # dias
			
			# Tiempo de ciclo
			t_cycle = Q_star / D * 365  # dias
			
			# Costo total
			total_cost = sqrt(2 * D * S * H * (P - D) / P)
			
			# Corridas por año
			runs_per_year = D / Q_star
			
			return (
			Q_star = Q_star,
			I_max = I_max,
			t_production = t_production,
			t_cycle = t_cycle,
			total_cost = total_cost,
			runs_per_year = runs_per_year
			)
			end
			
			# Ejemplo de aplicacion: Fabrica de componentes electronicos
			D_prod = 8000   # unidades/año
			S_prod = 100    # $ por preparacion
			H_prod = 2.5    # $ por unidad/año
			P_prod = 12000  # unidades/año
			
			resultado_prod = production_lot_model(D_prod, S_prod, H_prod, P_prod)
			
			println("MODELO DE LOTE DE PRODUCCION")
			println("="^50)
			@printf("Cantidad optima (Q*): %.2f unidades\n", 
			resultado_prod.Q_star)
			@printf("Inventario maximo: %.2f unidades\n", 
			resultado_prod.I_max)
			@printf("Tiempo de produccion: %.2f dias\n", 
			resultado_prod.t_production)
			@printf("Tiempo de ciclo: %.2f dias\n", 
			resultado_prod.t_cycle)
			@printf("Costo total anual: \$%.2f\n", 
			resultado_prod.total_cost)
			@printf("Corridas por año: %.2f\n", 
			resultado_prod.runs_per_year)
			
			# Analisis del ciclo de produccion
			println("\nANALISIS DEL CICLO:")
			@printf("Tiempo produciendo: %.1f%% del ciclo\n", 
			resultado_prod.t_production/resultado_prod.t_cycle * 100)
			@printf("Tiempo sin producir: %.1f%% del ciclo\n", 
			(1 - resultado_prod.t_production/resultado_prod.t_cycle) * 100)
		\end{lstlisting}
		
	\end{tcolorbox}
	
	\subsection{Comparaci\'on: EOQ vs Lote de Producci\'on}
	
	\begin{tcolorbox}[enhanced,colback=azulclaro,colframe=azulprincipal,boxrule=2pt,arc=8pt,
		title={\bfseries\color{white} \faBalanceScale\ COMPARACI\'ON DE MODELOS}]
		
		\begin{lstlisting}[language=Julia,basicstyle=\footnotesize\ttfamily]
			# Comparacion entre EOQ clasico y modelo de produccion
			function compare_eoq_production(D, S, H, P)
			# EOQ clasico (como si fuera compra instantanea)
			Q_eoq = sqrt(2 * D * S / H)
			cost_eoq = sqrt(2 * D * S * H)
			
			# Modelo de produccion
			result_prod = production_lot_model(D, S, H, P)
			
			# Calcular ahorros
			savings = cost_eoq - result_prod.total_cost
			savings_percent = (savings / cost_eoq) * 100
			
			println("COMPARACION: EOQ vs LOTE DE PRODUCCION")
			println("="^60)
			@printf("EOQ Clasico (Q): %.2f unidades\n", Q_eoq)
			@printf("Lote Produccion (Q*): %.2f unidades\n", result_prod.Q_star)
			@printf("Diferencia en cantidad: %.2f unidades\n", 
			result_prod.Q_star - Q_eoq)
			println("-"^60)
			@printf("Costo EOQ: \$%.2f\n", cost_eoq)
			@printf("Costo Produccion: \$%.2f\n", result_prod.total_cost)
			@printf("Ahorro anual: \$%.2f (%.1f%%)\n", savings, savings_percent)
			
			return (
			eoq = Q_eoq,
			production = result_prod.Q_star,
			savings = savings,
			savings_percent = savings_percent
			)
			end
			
			# Ejecutar comparacion con datos del ejemplo
			comparacion = compare_eoq_production(D_prod, S_prod, H_prod, P_prod)
		\end{lstlisting}
		
	\end{tcolorbox}
	
	\section{Modelo con Tiempo de Entrega}
	
	\begin{tcolorbox}[enhanced,colback=naranjaclaro,colframe=naranjaacento,boxrule=3pt,arc=12pt,
		drop shadow,title={\Large\bfseries\color{white} \faHourglass\ MODELO CON LEAD TIME},breakable]
		
		\begin{lstlisting}[language=Julia,basicstyle=\footnotesize\ttfamily]
			"""
			EOQ con tiempo de entrega
			Parametros adicionales:
			- L: Tiempo de entrega en dias
			- working_days: Dias laborables por año
			"""
			function eoq_lead_time(D, S, H, L, working_days=365)
			# EOQ basico
			Q_star = sqrt(2 * D * S / H)
			
			# Punto de reorden
			daily_demand = D / working_days
			R = daily_demand * L
			
			# Tiempo entre pedidos
			time_between_orders = Q_star / daily_demand
			
			# Numero de pedidos por año
			orders_per_year = D / Q_star
			
			# Costo total
			total_cost = sqrt(2 * D * S * H)
			
			return (
			Q_star = Q_star,
			R = R,
			daily_demand = daily_demand,
			time_between_orders = time_between_orders,
			orders_per_year = orders_per_year,
			total_cost = total_cost
			)
			end
			
			# Ejemplo: ICR LLC
			D_icr = 19500    # componentes/año
			S_icr = 50       # $ por pedido
			unit_cost_icr = 22.00
			H_icr = 0.02 * 12 * unit_cost_icr  # 5.28 $/componente/año
			L_icr = 4        # dias
			working_days_icr = 307
			
			resultado_icr = eoq_lead_time(D_icr, S_icr, H_icr, L_icr, working_days_icr)
			
			println("ICR LLC - EOQ CON TIEMPO DE ENTREGA")
			println("="^60)
			@printf("Cantidad optima (Q*): %.0f componentes\n", 
			resultado_icr.Q_star)
			@printf("Punto de reorden (R): %.0f componentes\n", 
			resultado_icr.R)
			@printf("Demanda diaria: %.2f componentes/dia\n", 
			resultado_icr.daily_demand)
			@printf("Tiempo entre pedidos: %.2f dias\n", 
			resultado_icr.time_between_orders)
			@printf("Pedidos por año: %.1f\n", 
			resultado_icr.orders_per_year)
			@printf("Costo total anual: \$%.2f\n", 
			resultado_icr.total_cost)
		\end{lstlisting}
		
	\end{tcolorbox}
	
	\section{Modelo con Descuentos por Cantidad}
	
	\begin{tcolorbox}[enhanced,colback=violetaclaro,colframe=moradoacento,boxrule=3pt,arc=12pt,
		drop shadow,title={\Large\bfseries\color{white} \faTag\ DESCUENTOS POR CANTIDAD},breakable]
		
		\begin{lstlisting}[language=Julia,basicstyle=\footnotesize\ttfamily]
			"""
			Modelo EOQ con descuentos por cantidad
			"""
			struct DiscountTier
			min_qty::Float64
			max_qty::Float64
			unit_price::Float64
			end
			
			function eoq_quantity_discount(D, S, i, discount_tiers::Vector{DiscountTier})
			results = []
			
			for tier in discount_tiers
			H = i * tier.unit_price  # Costo de almacenamiento
			
			# EOQ para este nivel de precio
			Q_eoq = sqrt(2 * D * S / H)
			
			# Verificar si Q_eoq esta en el rango valido
			if tier.min_qty <= Q_eoq <= tier.max_qty
			Q_feasible = Q_eoq
			else
			# Usar el punto de quiebre mas cercano
			Q_feasible = tier.min_qty
			end
			
			# Calcular costo total
			annual_purchase_cost = D * tier.unit_price
			annual_ordering_cost = (D / Q_feasible) * S
			annual_holding_cost = (Q_feasible / 2) * H
			total_cost = annual_purchase_cost + annual_ordering_cost + annual_holding_cost
			
			push!(results, (
			tier = tier,
			Q_eoq = Q_eoq,
			Q_feasible = Q_feasible,
			total_cost = total_cost,
			purchase_cost = annual_purchase_cost,
			ordering_cost = annual_ordering_cost,
			holding_cost = annual_holding_cost
			))
			end
			
			# Encontrar la opcion de menor costo
			min_cost_idx = argmin([r.total_cost for r in results])
			optimal_result = results[min_cost_idx]
			
			return (
			optimal = optimal_result,
			all_options = results
			)
			end
			
			# Datos del problema MBI Computadoras
			D_mbi = 5200  # discos/año
			S_mbi = 50    # $ por pedido
			i_mbi = 0.20  # 20% tasa de costo de mantener
			
			# Definir estructura de descuentos
			descuentos_mbi = [
			DiscountTier(1, 99, 100.0),
			DiscountTier(100, 499, 95.0),
			DiscountTier(500, Inf, 90.0)
			]
			
			resultado_mbi = eoq_quantity_discount(D_mbi, S_mbi, i_mbi, descuentos_mbi)
			
			println("MBI COMPUTADORAS - DESCUENTOS POR CANTIDAD")
			println("="^70)
			
			println("ANALISIS DE TODAS LAS OPCIONES:")
			for (i, option) in enumerate(resultado_mbi.all_options)
			tier = option.tier
			println("\nCategoria $i: $(tier.min_qty) - $(tier.max_qty == Inf ? "Infinito" : tier.max_qty)")
			@printf("  Precio unitario: \$%.2f\n", tier.unit_price)
			@printf("  EOQ calculado: %.2f\n", option.Q_eoq)
			@printf("  Cantidad factible: %.0f\n", option.Q_feasible)
			@printf("  Costo total: \$%.2f\n", option.total_cost)
			end
			
			println("\n" * "="^70)
			println("SOLUCION OPTIMA:")
			optimal = resultado_mbi.optimal
			@printf("Cantidad optima: %.0f unidades\n", optimal.Q_feasible)
			@printf("Precio unitario: \$%.2f\n", optimal.tier.unit_price)
			@printf("Costo total anual: \$%.2f\n", optimal.total_cost)
		\end{lstlisting}
		
	\end{tcolorbox}
	
	\section{Modelos Probabil\'isticos}
	
	\begin{tcolorbox}[enhanced,colback=naranjaclaro,colframe=naranjaacento,boxrule=3pt,arc=12pt,
		drop shadow,title={\Large\bfseries\color{white} \faChartLine\ MODELOS PROBABIL\'ISTICOS},breakable]
		
		\begin{lstlisting}[language=Julia,basicstyle=\footnotesize\ttfamily]
			"""
			Modelo de demanda discreta (newsvendor problem)
			"""
			function newsvendor_discrete(demand_probs::Dict, c, h, p)
			# c: costo de compra
			# h: costo de almacenamiento 
			# p: costo de escasez (perdida por faltante)
			
			# Calcular proporcion critica
			critical_ratio = (p - c) / (p + h)
			
			# Ordenar demandas
			demands = sort(collect(keys(demand_probs)))
			
			# Calcular probabilidades acumuladas
			cum_prob = 0.0
			optimal_q = demands[1]
			
			println("Analisis del modelo newsvendor:")
			println("Proporcion critica: $(round(critical_ratio, digits=4))")
			println("\nTabla de probabilidades:")
			println("Demanda\tProb\tProb_Acum")
			
			for d in demands
			cum_prob += demand_probs[d]
			println("$d\t$(demand_probs[d])\t$(round(cum_prob, digits=3))")
			
			if cum_prob >= critical_ratio && optimal_q == demands[1]
			optimal_q = d
			end
			end
			
			return (
			optimal_quantity = optimal_q,
			critical_ratio = critical_ratio,
			demands = demands,
			probabilities = demand_probs
			)
			end
			
			# Ejemplo: Producto con demanda incierta
			demanda_probs = Dict(
			0 => 0.05, 1 => 0.10, 2 => 0.10, 3 => 0.20, 4 => 0.25,
			5 => 0.15, 6 => 0.05, 7 => 0.05, 8 => 0.05
			)
			
			c_producto = 10  # costo de compra
			h_producto = 1   # costo de almacenamiento
			p_producto = 15  # costo de escasez
			
			resultado_newsvendor = newsvendor_discrete(demanda_probs, c_producto, 
			h_producto, p_producto)
			
			println("\nPRODUCTO CON DEMANDA INCIERTA")
			println("="^50)
			@printf("Cantidad optima a ordenar: %d unidades\n", 
			resultado_newsvendor.optimal_quantity)
		\end{lstlisting}
		
	\end{tcolorbox}
	
	\subsection{Modelo Probabil\'istico de Chicago Cheese}
	
	\begin{tcolorbox}[enhanced,colback=verdeclaro,colframe=verdeprincipal,boxrule=2pt,arc=8pt,
		title={\bfseries\color{white} \faCode\ CHICAGO CHEESE - JULIA},breakable]
		
		\begin{lstlisting}[language=Julia,basicstyle=\footnotesize\ttfamily]
			"""
			Modelo probabilistico de un periodo - Chicago Cheese
			"""
			function chicago_cheese_model()
			# Datos del problema
			selling_price = 100  # precio de venta por caja
			production_cost = 75 # costo de produccion por caja
			salvage_value = 50   # valor de salvamento por caja
			
			# Costos marginales
			Co = selling_price - production_cost  # costo de oportunidad (falta)
			Cu = production_cost - salvage_value  # costo de exceso
			
			# Proporcion critica
			critical_ratio = Co / (Co + Cu)
			
			# Distribucion de demanda
			demands = [10, 11, 12, 13, 14]
			probabilities = [0.2, 0.3, 0.2, 0.2, 0.1]
			cum_probabilities = cumsum(probabilities)
			
			println("CHICAGO CHEESE - ANALISIS DE PRODUCCION")
			println("="^60)
			@printf("Precio de venta: \$%d por caja\n", selling_price)
			@printf("Costo de produccion: \$%d por caja\n", production_cost)
			@printf("Valor de salvamento: \$%d por caja\n", salvage_value)
			println("-"^60)
			@printf("Costo de oportunidad (Co): \$%d\n", Co)
			@printf("Costo de exceso (Cu): \$%d\n", Cu)
			@printf("Proporcion critica: %.3f\n", critical_ratio)
			
			println("\nDistribucion de demanda:")
			println("Demanda\tProb\tProb_Acum")
			
			optimal_production = demands[1]
			for (i, d) in enumerate(demands)
			println("$d\t$(probabilities[i])\t$(cum_probabilities[i])")
			
			if cum_probabilities[i] >= critical_ratio && optimal_production == demands[1]
			optimal_production = d
			end
			end
			
			println("\n" * "="^60)
			println("RECOMENDACION:")
			println("Producir 11 o 12 cajas de queso")
			println("Como la proporcion critica es 0.5 y P(D <= 11) = 0.5")
			println("La decision optima puede ser 11 o 12 cajas")
			
			return (
			optimal_production = optimal_production,
			critical_ratio = critical_ratio,
			Co = Co,
			Cu = Cu
			)
			end
			
			# Ejecutar analisis
			resultado_chicago = chicago_cheese_model()
		\end{lstlisting}
		
	\end{tcolorbox}
	
	\section{Modelo de Producci\'on con Pedidos Atrasados}
	
	\begin{tcolorbox}[enhanced,colback=violetaclaro,colframe=moradoacento,boxrule=3pt,arc=12pt,
		drop shadow,title={\Large\bfseries\color{white} \faIndustry\ PRODUCCI\'ON CON BACKORDERS},breakable]
		
		\begin{lstlisting}[language=Julia,basicstyle=\footnotesize\ttfamily]
			"""
			Modelo de produccion con pedidos atrasados permitidos
			Combina el modelo de lote de produccion con la posibilidad de 
			tener demanda pendiente (backorders)
			
			Parametros:
			- D: Demanda anual
			- S: Costo de preparacion por corrida
			- H: Costo de almacenamiento por unidad por año
			- P: Tasa de produccion anual
			- B: Costo de deficit por unidad por año
			"""
			function production_backorder_model(D, S, H, P, B)
			if P <= D
			error("La tasa de produccion debe ser mayor que la demanda")
			end
			
			# PASO 1: Calcular la cantidad optima de produccion
			# Formula: Q* = sqrt(2DS/H) * sqrt((H+B)/B) * sqrt(P/(P-D))
			Q_base = sqrt(2 * D * S / H)  # EOQ basico
			factor_backorder = sqrt((H + B) / B)  # Factor por backorders
			factor_production = sqrt(P / (P - D))  # Factor por produccion
			
			Q_star = Q_base * factor_backorder * factor_production
			
			# PASO 2: Calcular el inventario maximo
			# Durante la produccion, el inventario maximo sera menor que Q*
			# porque parte de la produccion satisface la demanda inmediata
			I_max = Q_star * (B / (H + B)) * (1 - D/P)
			
			# PASO 3: Calcular el deficit maximo
			# Es la cantidad de demanda pendiente antes de iniciar produccion
			deficit_max = Q_star * (H / (H + B))
			
			# PASO 4: Tiempos del ciclo
			# Tiempo total del ciclo
			t_cycle = Q_star / D * 365  # dias
			
			# Tiempo de produccion
			t_production = Q_star / P * 365  # dias
			
			# Tiempo sin producir
			t_no_production = t_cycle - t_production
			
			# PASO 5: Costo total anual
			# Formula: CT = sqrt(2*D*S*H*B*(P-D)/(P*(H+B)))
			total_cost = sqrt(2 * D * S * H * B * (P - D) / (P * (H + B)))
			
			# PASO 6: Numero de corridas por año
			runs_per_year = D / Q_star
			
			return (
			Q_star = Q_star,
			I_max = I_max,
			deficit_max = deficit_max,
			t_cycle = t_cycle,
			t_production = t_production,
			t_no_production = t_no_production,
			total_cost = total_cost,
			runs_per_year = runs_per_year,
			# Informacion adicional para analisis
			Q_base = Q_base,
			factor_backorder = factor_backorder,
			factor_production = factor_production
			)
			end
		\end{lstlisting}
		
	\end{tcolorbox}
	
	\subsection{Ejercicio Resuelto: Empresa Ladrillera}
	
	\begin{tcolorbox}[enhanced,colback=naranjaclaro,colframe=naranjaacento,boxrule=2pt,arc=8pt,
		drop shadow,title={\bfseries\color{white} \faPuzzlePiece\ PROBLEMA EMPRESA LADRILLERA}]
		
		\textbf{Empresa Ladrillera:} Una empresa tiene demanda anual de 210,000 ladrillos. Los produce a ritmo mensual de 37,500 ladrillos. Costo de preparaci\'on \$450 por corrida. Costo anual de almacenamiento 1.2 \$/unidad. Costo anual por demanda pendiente 0.5 \$/unidad. Considerar 360 d\'ias/a\~no.
		
	\end{tcolorbox}
	
	\begin{tcolorbox}[enhanced,colback=azulclaro,colframe=azulprincipal,boxrule=2pt,arc=8pt,
		title={\bfseries\color{white} \faCode\ SOLUCI\'ON PASO A PASO},breakable]
		
		\begin{lstlisting}[language=Julia,basicstyle=\footnotesize\ttfamily]
			# DATOS DEL PROBLEMA
			D_ladrillos = 210000    # ladrillos/año
			P_ladrillos = 37500 * 12  # 37,500/mes = 450,000/año
			S_ladrillos = 450       # $ por corrida
			H_ladrillos = 1.2       # $ por ladrillo/año
			B_ladrillos = 0.5       # $ por ladrillo/año (deficit)
			
			println("EMPRESA LADRILLERA - PRODUCCION CON BACKORDERS")
			println("="^70)
			println("DATOS DEL PROBLEMA:")
			@printf("Demanda anual (D): %d ladrillos\n", D_ladrillos)
			@printf("Tasa de produccion (P): %d ladrillos/año\n", P_ladrillos)
			@printf("Costo de preparacion (S): \$%.0f\n", S_ladrillos)
			@printf("Costo de almacenamiento (H): \$%.1f/ladrillo/año\n", H_ladrillos)
			@printf("Costo de deficit (B): \$%.1f/ladrillo/año\n", B_ladrillos)
			
			# RESOLVER EL PROBLEMA
			resultado_ladrillos = production_backorder_model(
			D_ladrillos, S_ladrillos, H_ladrillos, P_ladrillos, B_ladrillos)
			
			println("\n" * "="^70)
			println("ANALISIS PASO A PASO:")
			println("-"^70)
			
			println("PASO 1: Factores de calculo")
			@printf("EOQ basico: %.2f ladrillos\n", resultado_ladrillos.Q_base)
			@printf("Factor backorder: %.4f\n", resultado_ladrillos.factor_backorder)
			@printf("Factor produccion: %.4f\n", resultado_ladrillos.factor_production)
			
			println("\nPASO 2: Resultados principales")
			@printf("Cantidad optima (Q*): %.2f ladrillos\n", resultado_ladrillos.Q_star)
			@printf("Inventario maximo: %.2f ladrillos\n", resultado_ladrillos.I_max)
			@printf("Deficit maximo: %.2f ladrillos\n", resultado_ladrillos.deficit_max)
			
			println("\nPASO 3: Analisis temporal")
			@printf("Tiempo de ciclo: %.2f dias\n", resultado_ladrillos.t_cycle)
			@printf("Tiempo de produccion: %.2f dias\n", resultado_ladrillos.t_production)
			@printf("Tiempo sin producir: %.2f dias\n", resultado_ladrillos.t_no_production)
			
			println("\nPASO 4: Analisis economico")
			@printf("Costo total anual: \$%.2f\n", resultado_ladrillos.total_cost)
			@printf("Corridas por año: %.2f\n", resultado_ladrillos.runs_per_year)
			
			# COMPARACION CON MODELO SIN BACKORDERS
			resultado_sin_backorders = production_lot_model(
			D_ladrillos, S_ladrillos, H_ladrillos, P_ladrillos)
			
			ahorro = resultado_sin_backorders.total_cost - resultado_ladrillos.total_cost
			ahorro_porcentaje = (ahorro / resultado_sin_backorders.total_cost) * 100
			
			println("\nPASO 5: Comparacion con modelo sin backorders")
			@printf("Costo sin backorders: \$%.2f\n", resultado_sin_backorders.total_cost)
			@printf("Costo con backorders: \$%.2f\n", resultado_ladrillos.total_cost)
			@printf("Ahorro anual: \$%.2f (%.2f%%)\n", ahorro, ahorro_porcentaje)
		\end{lstlisting}
		
	\end{tcolorbox}
	
	\section{Modelo EOQ con Desabastecimientos Planeados}
	
	\begin{tcolorbox}[enhanced,colback=verdeclaro,colframe=verdeprincipal,boxrule=3pt,arc=12pt,
		drop shadow,title={\Large\bfseries\color{white} \faExclamationTriangle\ EOQ CON ESCASEZ PLANEADA},breakable]
		
		\begin{lstlisting}[language=Julia,basicstyle=\footnotesize\ttfamily]
			"""
			Modelo EOQ con desabastecimientos planeados
			Permite escasez planificada para reducir costos totales
			
			Parametros:
			- D: Demanda anual
			- S: Costo de pedido
			- H: Costo de almacenamiento por unidad por año
			- B: Costo de desabastecimiento por unidad por año
			- L: Tiempo de entrega (opcional)
			"""
			function eoq_planned_shortage(D, S, H, B, L=0)
			
			# PASO 1: Verificar que el modelo tenga sentido economico
			if B <= H
			println("ADVERTENCIA: El costo de escasez (B=$B) debe ser mayor")
			println("que el costo de almacenamiento (H=$H) para que tenga")
			println("sentido economico permitir desabastecimientos.")
			end
			
			# PASO 2: Calcular cantidad optima de pedido
			# Formula: Q* = sqrt(2*D*S/H) * sqrt((H+B)/B)
			Q_eoq_basico = sqrt(2 * D * S / H)
			factor_escasez = sqrt((H + B) / B)
			Q_star = Q_eoq_basico * factor_escasez
			
			# PASO 3: Calcular inventario maximo
			# Es la cantidad maxima en stock antes de agotar
			I_max = Q_star * (B / (H + B))
			
			# PASO 4: Calcular desabastecimiento maximo
			# Es la cantidad maxima de demanda pendiente
			S_max = Q_star * (H / (H + B))
			
			# PASO 5: Calcular tiempos del ciclo
			tiempo_ciclo = Q_star / D * 365  # dias
			tiempo_con_inventario = I_max / D * 365  # dias
			tiempo_desabastecido = S_max / D * 365  # dias
			
			# PASO 6: Calcular costos
			# Costo total: CT = sqrt(2*D*S*H*B/(H+B))
			costo_total = sqrt(2 * D * S * H * B / (H + B))
			
			# Costo de pedidos
			costo_pedidos = (D / Q_star) * S
			
			# Costo de almacenamiento
			costo_almacenamiento = (I_max^2 / (2 * Q_star)) * H
			
			# Costo de desabastecimiento
			costo_desabastecimiento = (S_max^2 / (2 * Q_star)) * B
			
			# PASO 7: Punto de reorden (si hay tiempo de entrega)
			if L > 0
			demanda_diaria = D / 365
			R = demanda_diaria * L - S_max  # Punto de reorden negativo!
			else
			R = -S_max  # Sin tiempo de entrega
			end
			
			# PASO 8: Numero de pedidos por año
			pedidos_por_año = D / Q_star
			
			return (
			# Resultados principales
			Q_star = Q_star,
			I_max = I_max,
			S_max = S_max,
			R = R,
			
			# Analisis temporal
			tiempo_ciclo = tiempo_ciclo,
			tiempo_con_inventario = tiempo_con_inventario,
			tiempo_desabastecido = tiempo_desabastecido,
			
			# Analisis de costos
			costo_total = costo_total,
			costo_pedidos = costo_pedidos,
			costo_almacenamiento = costo_almacenamiento,
			costo_desabastecimiento = costo_desabastecimiento,
			
			# Metricas operativas
			pedidos_por_año = pedidos_por_año,
			
			# Para comparaciones
			Q_eoq_basico = Q_eoq_basico,
			factor_escasez = factor_escasez
			)
			end
		\end{lstlisting}
		
	\end{tcolorbox}
	
	\subsection{Ejercicio Resuelto: Producto con Escasez Planeada}
	
	\begin{tcolorbox}[enhanced,colback=naranjaclaro,colframe=naranjaacento,boxrule=2pt,arc=8pt,
		drop shadow,title={\bfseries\color{white} \faPuzzlePiece\ PROBLEMA CON DESABASTECIMIENTO}]
		
		\textbf{Producto con Escasez:} Un administrador considera un producto con demanda de 800 unidades anuales. Costo de mantener 0.25€/unidad/mes. Costo de pedido 150€. Costo de no tener unidad 20€/unidad/año. Tiempo de entrega 1 mes.
		
	\end{tcolorbox}
	
	\begin{tcolorbox}[enhanced,colback=azulclaro,colframe=azulprincipal,boxrule=2pt,arc=8pt,
		title={\bfseries\color{white} \faCode\ SOLUCI\'ON DETALLADA},breakable]
		
		\begin{lstlisting}[language=Julia,basicstyle=\footnotesize\ttfamily]
			# DATOS DEL PROBLEMA
			D_producto = 800          # unidades/año
			S_producto = 150          # € por pedido  
			H_producto = 0.25 * 12    # 0.25€/mes = 3€/año por unidad
			B_producto = 20           # € por unidad/año
			L_producto = 1/12         # 1 mes = 1/12 años
			
			println("PRODUCTO CON ESCASEZ PLANEADA")
			println("="^60)
			println("ANALISIS PASO A PASO:")
			println("-"^60)
			
			# PASO 1: Mostrar datos y verificar viabilidad
			println("PASO 1: Datos del problema")
			@printf("Demanda anual: %d unidades\n", D_producto)
			@printf("Costo de pedido: %.0f €\n", S_producto)
			@printf("Costo de almacenamiento: %.2f €/unidad/año\n", H_producto)
			@printf("Costo de desabastecimiento: %.0f €/unidad/año\n", B_producto)
			@printf("Tiempo de entrega: %.3f años (1 mes)\n", L_producto)
			
			if B_producto > H_producto
			println("VIABLE: Costo escasez > Costo almacenamiento")
			else
			println("REVISAR: Costo escasez <= Costo almacenamiento")
			end
			
			# PASO 2: Resolver con el modelo
			resultado_escasez = eoq_planned_shortage(
			D_producto, S_producto, H_producto, B_producto, L_producto)
			
			println("\nPASO 2: Calculos principales")
			@printf("EOQ basico (sin escasez): %.2f unidades\n", 
			resultado_escasez.Q_eoq_basico)
			@printf("Factor de escasez: %.4f\n", resultado_escasez.factor_escasez)
			@printf("Cantidad optima (Q*): %.2f unidades\n", resultado_escasez.Q_star)
			
			println("\nPASO 3: Niveles de inventario")
			@printf("Inventario maximo: %.2f unidades\n", resultado_escasez.I_max)
			@printf("Desabastecimiento maximo: %.2f unidades\n", resultado_escasez.S_max)
			@printf("Punto de reorden: %.2f unidades\n", resultado_escasez.R)
			
			println("\nPASO 4: Analisis temporal (dias)")
			@printf("Tiempo de ciclo: %.2f dias\n", resultado_escasez.tiempo_ciclo)
			@printf("Tiempo con inventario: %.2f dias\n", 
			resultado_escasez.tiempo_con_inventario)
			@printf("Tiempo desabastecido: %.2f dias\n", 
			resultado_escasez.tiempo_desabastecido)
			
			println("\nPASO 5: Analisis de costos (€)")
			@printf("Costo de pedidos: %.2f\n", resultado_escasez.costo_pedidos)
			@printf("Costo de almacenamiento: %.2f\n", 
			resultado_escasez.costo_almacenamiento)
			@printf("Costo de desabastecimiento: %.2f\n", 
			resultado_escasez.costo_desabastecimiento)
			@printf("COSTO TOTAL: %.2f €\n", resultado_escasez.costo_total)
			
			# PASO 6: Comparacion con EOQ clasico
			costo_eoq_clasico = sqrt(2 * D_producto * S_producto * H_producto)
			ahorro_anual = costo_eoq_clasico - resultado_escasez.costo_total
			porcentaje_ahorro = (ahorro_anual / costo_eoq_clasico) * 100
			
			println("\nPASO 6: Comparacion con EOQ clasico")
			@printf("Costo EOQ clasico: %.2f €\n", costo_eoq_clasico)
			@printf("Costo con escasez: %.2f €\n", resultado_escasez.costo_total)
			@printf("Ahorro anual: %.2f € (%.2f%%)\n", ahorro_anual, porcentaje_ahorro)
			
			println("\nPASO 7: Interpretacion gerencial")
			println("- El modelo permite desabastecimientos controlados")
			println("- Se ahorra en costos de almacenamiento")
			@printf("   - %.1f%% del tiempo habra inventario disponible\n", 
			(resultado_escasez.tiempo_con_inventario/resultado_escasez.tiempo_ciclo)*100)
			@printf("   - %.1f%% del tiempo habra desabastecimiento\n",
			(resultado_escasez.tiempo_desabastecido/resultado_escasez.tiempo_ciclo)*100)
		\end{lstlisting}
		
	\end{tcolorbox}
	
	\section{Modelo de Revisi\'on Peri\'odica (R,S)}
	
	\begin{tcolorbox}[enhanced,colback=violetaclaro,colframe=moradoacento,boxrule=3pt,arc=12pt,
		drop shadow,title={\Large\bfseries\color{white} \faCalendar\ MODELO (R,S) DE REVISION PERIODICA},breakable]
		
		\begin{lstlisting}[language=Julia,basicstyle=\footnotesize\ttfamily]
			"""
			Modelo de revision periodica (R,S)
			Sistema donde se revisa el inventario cada R periodos
			y se ordena hasta llegar al nivel S
			
			Parametros:
			- demand_probs: Diccionario con probabilidades de demanda
			- h: Costo de almacenamiento por unidad por periodo
			- p: Costo de ruptura/faltante por unidad
			- review_period: Periodo de revision R
			"""
			function periodic_review_model(demand_probs::Dict, h, p, review_period=1)
			
			# PASO 1: Calcular la proporcion critica
			# Formula: p/(h+p) donde p es costo de ruptura, h costo almacenamiento
			critical_ratio = p / (h + p)
			
			println("MODELO DE REVISION PERIODICA (R,S)")
			println("="^60)
			println("PASO 1: Calculo de proporcion critica")
			println("Costo de almacenamiento (h): $(h)")
			println("Costo de ruptura (p): $(p)")
			println("Proporcion critica: p/(h+p) = $(p)/($(p)+$(h)) = $(critical_ratio)")
			
			# PASO 2: Crear tabla de probabilidades acumuladas
			demandas = sort(collect(keys(demand_probs)))
			
			println("\nPASO 2: Tabla de analisis de demanda")
			println("d\tP(d)\tP(D<=d)\tp(d)/d\tSuma q*p(d)/d\tM(D<=d)")
			println("-"^60)
			
			prob_acum = 0.0
			suma_q_prob_d = 0.0
			tabla_resultados = []
			
			for d in demandas
			prob_d = demand_probs[d]
			prob_acum += prob_d
			
			# Calcular p(d)/d (costo marginal por unidad)
			if d > 0
			marginal_cost = prob_d / d
			suma_q_prob_d += d * marginal_cost
			else
			marginal_cost = Inf  # Evitar division por cero
			suma_q_prob_d = 0.0
			end
			
			# Funcion M(D<=d) para toma de decisiones
			M_d = prob_acum + (1/2) * suma_q_prob_d
			
			push!(tabla_resultados, (
			demanda = d,
			prob = prob_d,
			prob_acum = prob_acum,
			marginal_cost = marginal_cost,
			suma_marginal = suma_q_prob_d,
			M_value = M_d
			))
			
			println("$d\t$(prob_d)\t$(prob_acum)\t", end="")
			if marginal_cost == Inf
			println("Infinito\t$(suma_q_prob_d)\t$(M_d)")
			else
			println("$(marginal_cost)\t$(suma_q_prob_d)\t$(M_d)")
			end
			end
			
			# PASO 3: Encontrar el nivel optimo S
			optimal_S = demandas[1]  # Valor por defecto
			
			for resultado in tabla_resultados
			if resultado.prob_acum >= critical_ratio
			optimal_S = resultado.demanda
			break
			end
			end
			
			println("\nPASO 3: Determinacion del nivel optimo S")
			println("Buscamos el primer d donde P(D<=d) >= $(critical_ratio)")
			println("Nivel optimo S* = $(optimal_S) unidades")
			
			# PASO 4: Calcular metricas de desempeño
			# Calcular la demanda esperada
			demanda_esperada = sum(d * demand_probs[d] for d in demandas)
			
			# Calcular nivel de servicio
			prob_no_ruptura = sum(demand_probs[d] for d in demandas if d <= optimal_S)
			nivel_servicio = prob_no_ruptura * 100
			
			# Inventario promedio
			inventario_promedio = optimal_S - demanda_esperada
			
			println("\nPASO 4: Metricas de desempeño")
			println("Demanda esperada: $(demanda_esperada) unidades")
			println("Inventario promedio: $(inventario_promedio) unidades")
			println("Nivel de servicio: $(nivel_servicio)%")
			println("Probabilidad de ruptura: $(100 - nivel_servicio)%")
			
			return (
			optimal_S = optimal_S,
			critical_ratio = critical_ratio,
			demand_expected = demanda_esperada,
			average_inventory = inventario_promedio,
			service_level = nivel_servicio,
			tabla_analisis = tabla_resultados,
			review_period = review_period
			)
			end
		\end{lstlisting}
		
	\end{tcolorbox}
	
	\subsection{Ejercicio Resuelto: Producto con Revisi\'on Peri\'odica}
	
	\begin{tcolorbox}[enhanced,colback=naranjaclaro,colframe=naranjaacento,boxrule=2pt,arc=8pt,
		drop shadow,title={\bfseries\color{white} \faPuzzlePiece\ PROBLEMA DE REVISI\'ON PERI\'ODICA}]
		
		\textbf{Producto con Revisi\'on:} Un producto cuesta 60€, pero si no se dispone cuando se necesita, ocasiona pérdida de 800€. Si se ha comprado y no se utiliza, debe pagarse almacenamiento de 10€ por período. La demanda es discreta con probabilidades dadas en tabla.
		
	\end{tcolorbox}
	
	\begin{tcolorbox}[enhanced,colback=azulclaro,colframe=azulprincipal,boxrule=2pt,arc=8pt,
		title={\bfseries\color{white} \faCode\ SOLUCI\'ON COMPLETA},breakable]
		
		\begin{lstlisting}[language=Julia,basicstyle=\footnotesize\ttfamily]
			# DATOS DEL PROBLEMA (basados en Tabla 8.4 del PDF)
			# Probabilidades de demanda discreta  
			demanda_revision = Dict(
			0 => 0.1,   # P(D=0) = 0.1
			1 => 0.2,   # P(D=1) = 0.2  
			2 => 0.2,   # P(D=2) = 0.2
			3 => 0.3,   # P(D=3) = 0.3
			4 => 0.1,   # P(D=4) = 0.1
			5 => 0.1    # P(D=5) = 0.1
			)
			
			# Costos del problema
			costo_ruptura = 800.0      # € por unidad faltante
			costo_almacenamiento = 10.0 # € por unidad en inventario por periodo
			
			println("PRODUCTO CON REVISION PERIODICA - ANALISIS COMPLETO")
			println("="^70)
			
			# RESOLVER EL MODELO
			resultado_revision = periodic_review_model(
			demanda_revision, costo_almacenamiento, costo_ruptura)
			
			println("\nRESUMEN EJECUTIVO:")
			println("="^50)
			println("NIVEL OPTIMO de inventario (S*): $(resultado_revision.optimal_S) unidades")
			println("PROPORCION critica: $(resultado_revision.critical_ratio)")
			println("DEMANDA esperada: $(resultado_revision.demand_expected) unidades")
			println("INVENTARIO promedio: $(resultado_revision.average_inventory) unidades")
			println("NIVEL de servicio: $(resultado_revision.service_level)%")
			
			println("\nINTERPRETACION GERENCIAL:")
			println("-"^50)
			println("1. POLITICA OPTIMA:")
			@printf("   - Revisar inventario cada periodo\n")
			@printf("   - Ordenar hasta tener %d unidades\n", resultado_revision.optimal_S)
			
			println("\n2. IMPLICACIONES OPERATIVAS:")
			println("   - Costo muy alto de ruptura (€$(costo_ruptura)) justifica inventario alto")
			println("   - Se mantendra inventario promedio de $(resultado_revision.average_inventory) unidades")
			println("   - $(resultado_revision.service_level)% de las veces se satisfara la demanda completamente")
			
			println("\n3. ANALISIS DE SENSIBILIDAD:")
			# Probar diferentes niveles S para mostrar el impacto
			println("   Impacto de diferentes niveles S:")
			for S_test in 2:6
			prob_satisfaccion = sum(demanda_revision[d] for d in keys(demanda_revision) if d <= S_test)
			inventario_prom = S_test - resultado_revision.demand_expected
			println("   S=$(S_test): Servicio=$(round(prob_satisfaccion*100,digits=1))%, Inv.Prom=$(round(inventario_prom,digits=1))")
			end
			
			println("\n4. COMPARACION DE COSTOS ESPERADOS:")
			# Calcular costo esperado para el nivel optimo
			S_opt = resultado_revision.optimal_S
			costo_almacen_esp = resultado_revision.average_inventory * costo_almacenamiento
			
			# Calcular costo de ruptura esperado
			costo_ruptura_esp = 0.0
			for d in keys(demanda_revision)
			if d > S_opt
			faltante = d - S_opt
			costo_ruptura_esp += faltante * costo_ruptura * demanda_revision[d]
			end
			end
			
			costo_total_esp = costo_almacen_esp + costo_ruptura_esp
			
			println("   Costo almacenamiento esperado: €$(round(costo_almacen_esp,digits=2))")
			println("   Costo ruptura esperado: €$(round(costo_ruptura_esp,digits=2))")
			println("   COSTO TOTAL ESPERADO: €$(round(costo_total_esp,digits=2)) por periodo")
		\end{lstlisting}
		
	\end{tcolorbox}
	
	\section{Modelo con Demanda Normal}
	
	\begin{tcolorbox}[enhanced,colback=azulclaro,colframe=azulprincipal,boxrule=2pt,arc=8pt,
		title={\bfseries\color{white} \faCode\ DEMANDA NORMAL - TIENDA DE ABARROTES},breakable]
		
		\begin{lstlisting}[language=Julia,basicstyle=\footnotesize\ttfamily]
			"""
			Modelo de inventario con demanda normal y backorders
			"""
			function inventory_normal_demand(mu0, sigma0, k, L, h, Cu, weeks_per_year=52)
			# Parametros de demanda durante lead time
			muL = mu0 * L / weeks_per_year
			sigmaL = sigma0 * sqrt(L / weeks_per_year)
			
			# Para encontrar Q* y R* optimos, usamos el metodo iterativo
			# Inicializacion
			Q = sqrt(2 * mu0 * k / h)  # EOQ inicial
			
			# Funcion para calcular el punto de reorden optimo
			function optimal_reorder_point(Q, muL, sigmaL, h, Cu)
			# Factor de seguridad optimo
			z_star = quantile(Normal(0,1), Cu / (Cu + h))
			R_star = muL + z_star * sigmaL
			return R_star, z_star
			end
			
			# Iteracion para encontrar Q* y R* simultaneamente
			for iteration in 1:10
			R, z = optimal_reorder_point(Q, muL, sigmaL, h, Cu)
			
			# Funcion de perdida unitaria esperada
			L_z = pdf(Normal(0,1), z) - z * (1 - cdf(Normal(0,1), z))
			
			# Actualizar Q
			Q_new = sqrt(2 * mu0 * (k + Cu * sigmaL * L_z) / h)
			
			if abs(Q_new - Q) < 0.01
			Q = Q_new
			break
			end
			Q = Q_new
			end
			
			R, z = optimal_reorder_point(Q, muL, sigmaL, h, Cu)
			
			return (
			Q_star = Q,
			R_star = R,
			muL = muL,
			sigmaL = sigmaL,
			z_star = z,
			orders_per_year = mu0 / Q
			)
			end
			
			# Datos del problema: Tienda de abarrotes
			mu0_tienda = 1000   # cajas/año (promedio)
			sigma0_tienda = 40.8   # cajas/año (desviacion estandar)
			k_tienda = 50      # $ por pedido
			L_tienda = 2       # semanas de lead time
			h_tienda = 10      # $ por caja/año
			Cu_tienda = 20     # $ por caja (costo de agotamiento)
			
			resultado_tienda = inventory_normal_demand(mu0_tienda, sigma0_tienda, k_tienda, 
			L_tienda, h_tienda, Cu_tienda)
			
			println("TIENDA DE ABARROTES - DEMANDA NORMAL")
			println("="^60)
			@printf("Demanda promedio anual: %.0f cajas\n", mu0_tienda)
			@printf("Desviacion estandar anual: %.1f cajas\n", sigma0_tienda)
			@printf("Lead time: %d semanas\n", L_tienda)
			println("-"^60)
			@printf("Demanda promedio durante lead time: %.1f cajas\n", resultado_tienda.muL)
			@printf("Desviacion estandar durante lead time: %.3f cajas\n", resultado_tienda.sigmaL)
			println("-"^60)
			@printf("Cantidad optima (Q*): %.2f cajas\n", resultado_tienda.Q_star)
			@printf("Punto de reorden (R*): %.2f cajas\n", resultado_tienda.R_star)
			@printf("Factor de seguridad (z*): %.3f\n", resultado_tienda.z_star)
			@printf("Pedidos por año: %.2f\n", resultado_tienda.orders_per_year)
		\end{lstlisting}
		
	\end{tcolorbox}
	
	\section{Visualizaci\'on de Resultados}
	
	\begin{tcolorbox}[enhanced,colback=naranjaclaro,colframe=naranjaacento,boxrule=2pt,arc=8pt,
		title={\bfseries\color{white} \faChartArea\ GR\'AFICOS Y VISUALIZACI\'ON},breakable]
		
		\begin{lstlisting}[language=Julia,basicstyle=\footnotesize\ttfamily]
			"""
			Funciones para visualizar los modelos de inventario
			"""
			using Plots
			plotly()  # Backend interactivo
			
			function plot_eoq_costs(D, S, H, Q_range=nothing)
			if Q_range === nothing
			Q_opt = sqrt(2*D*S/H)
			Q_range = range(Q_opt*0.1, Q_opt*3, length=100)
			end
			
			ordering_costs = [D*S/Q for Q in Q_range]
			holding_costs = [Q*H/2 for Q in Q_range]
			total_costs = ordering_costs .+ holding_costs
			
			Q_opt = sqrt(2*D*S/H)
			min_cost = sqrt(2*D*S*H)
			
			p = plot(Q_range, [ordering_costs holding_costs total_costs],
			labels=["Costo de Pedido" "Costo de Almacen" "Costo Total"],
			linewidth=2,
			xlabel="Cantidad de Pedido (Q)",
			ylabel="Costo Anual (\$)",
			title="Analisis de Costos EOQ",
			legend=:topright)
			
			# Marcar el punto optimo
			scatter!(p, [Q_opt], [min_cost], 
			markersize=8, 
			markercolor=:red,
			label="Q* Optimo ($(round(Int,Q_opt)))")
			
			return p
			end
			
			function plot_inventory_level(Q, D, L=0, periods=4)
			"""Graficar el nivel de inventario a traves del tiempo"""
			
			days_per_cycle = Q / D * 365
			total_days = periods * days_per_cycle
			
			time_points = Float64[]
			inventory_levels = Float64[]
			
			for period in 0:(periods-1)
			cycle_start = period * days_per_cycle
			
			# Inicio del ciclo con Q unidades
			push!(time_points, cycle_start)
			push!(inventory_levels, Q)
			
			# Final del ciclo con 0 unidades
			push!(time_points, cycle_start + days_per_cycle - 0.01)
			push!(inventory_levels, 0.01)
			end
			
			p = plot(time_points, inventory_levels,
			linewidth=2,
			xlabel="Tiempo (dias)",
			ylabel="Nivel de Inventario",
			title="Patron de Inventario EOQ",
			legend=false,
			color=:blue)
			
			# Agregar lineas de reorden si hay lead time
			if L > 0
			reorder_level = D * L / 365
			hline!(p, [reorder_level], 
			linestyle=:dash, 
			color=:red,
			label="Punto de Reorden")
			end
			
			return p
			end
			
			function compare_models_table()
			"""Crear tabla comparativa de los diferentes modelos"""
			
			modelos = ["EOQ Basico", "EOQ con Backorders", "Lote de Produccion", 
			"Con Lead Time", "Descuentos", "Probabilistico"]
			
			caracteristicas = ["Demanda constante", "Permite faltantes", 
			"Produccion gradual", "Tiempo de entrega",
			"Precios variables", "Demanda incierta"]
			
			aplicaciones = ["Compras simples", "Servicio flexible",
			"Manufactura", "Compras con demora",
			"Compras a granel", "Productos perecederos"]
			
			df_comparacion = DataFrame(
			Modelo = modelos,
			Caracteristica = caracteristicas,
			Aplicacion = aplicaciones
			)
			
			return df_comparacion
			end
			
			# Generar ejemplo de graficos
			println("Generando visualizaciones...")
			
			# Ejemplo con datos de la clinica
			D = 10000; S = 50; H = 4.5
			plot_clinica = plot_eoq_costs(D, S, H)
			
			# Patron de inventario
			Q_opt = sqrt(2*D*S/H)
			plot_pattern = plot_inventory_level(Q_opt, D)
			
			# Tabla comparativa
			tabla_comparacion = compare_models_table()
			println("\nCOMPARACION DE MODELOS DE INVENTARIO:")
			println(tabla_comparacion)
		\end{lstlisting}
		
	\end{tcolorbox}
	
	\section{Funciones de Utilidad y Herramientas}
	
	\begin{tcolorbox}[enhanced,colback=grisclaro,colframe=grisoScuro,boxrule=2pt,arc=8pt,
		title={\bfseries\color{white} \faTools\ HERRAMIENTAS ADICIONALES},breakable]
		
		\begin{lstlisting}[language=Julia,basicstyle=\footnotesize\ttfamily]
			"""
			Modulo completo para analisis de inventarios
			"""
			module InventoryModels
			
			using Optim, Distributions, DataFrames, Printf
			
			export EOQResult, solve_eoq, solve_eoq_backorders, 
			solve_production_lot, analyze_quantity_discounts
			
			struct EOQResult
			Q_star::Float64
			total_cost::Float64
			orders_per_year::Float64
			time_between_orders::Float64
			additional_info::Dict{String, Any}
			end
			
			function sensitivity_analysis(D, S, H, param_range=0.8:0.1:1.2)
			"""Analisis de sensibilidad para parametros EOQ"""
			
			results = DataFrame(
			D_factor = Float64[],
			S_factor = Float64[],
			H_factor = Float64[],
			Q_star = Float64[],
			Total_Cost = Float64[]
			)
			
			base_Q = sqrt(2*D*S/H)
			base_cost = sqrt(2*D*S*H)
			
			for d_factor in param_range
			for s_factor in param_range
			for h_factor in param_range
			new_D = D * d_factor
			new_S = S * s_factor
			new_H = H * h_factor
			
			new_Q = sqrt(2*new_D*new_S/new_H)
			new_cost = sqrt(2*new_D*new_S*new_H)
			
			push!(results, (d_factor, s_factor, h_factor, new_Q, new_cost))
			end
			end
			end
			
			return results
			end
			
			function monte_carlo_inventory(D_dist, S_dist, H_dist, n_simulations=1000)
			"""Simulacion Monte Carlo para analisis de incertidumbre"""
			
			results = Float64[]
			
			for i in 1:n_simulations
			D_sample = rand(D_dist)
			S_sample = rand(S_dist)
			H_sample = rand(H_dist)
			
			Q_sample = sqrt(2 * D_sample * S_sample / H_sample)
			push!(results, Q_sample)
			end
			
			return (
			mean_Q = mean(results),
			std_Q = std(results),
			quantiles = quantile(results, [0.05, 0.25, 0.5, 0.75, 0.95]),
			all_results = results
			)
			end
			
			end  # fin del modulo
			
			# Ejemplo de analisis Monte Carlo
			println("ANALISIS MONTE CARLO - INCERTIDUMBRE EN PARAMETROS")
			println("="^60)
			
			# Definir distribuciones de parametros (con incertidumbre)
			D_distribution = Normal(10000, 500)   # Demanda con incertidumbre
			S_distribution = Normal(50, 5)        # Costo de pedido con incertidumbre  
			H_distribution = Normal(4.5, 0.5)     # Costo de almacen con incertidumbre
			
			mc_result = InventoryModels.monte_carlo_inventory(D_distribution, S_distribution, 
			H_distribution, 5000)
			
			@printf("EOQ promedio: %.2f unidades\n", mc_result.mean_Q)
			@printf("Desviacion estandar: %.2f unidades\n", mc_result.std_Q)
			println("\nCuantiles:")
			println("5%: $(round(mc_result.quantiles[1], digits=2))")
			println("25%: $(round(mc_result.quantiles[2], digits=2))")
			println("50% (mediana): $(round(mc_result.quantiles[3], digits=2))")
			println("75%: $(round(mc_result.quantiles[4], digits=2))")
			println("95%: $(round(mc_result.quantiles[5], digits=2))")
		\end{lstlisting}
		
	\end{tcolorbox}
	
	\section{Conclusiones y Mejores Pr\'acticas}
	
	\begin{tcolorbox}[enhanced,colback=grisclaro,colframe=grisoScuro,boxrule=3pt,arc=15pt,
		drop shadow,title={\Large\bfseries\color{white} \faFlag\ S\'INTESIS Y RECOMENDACIONES}]
		
		\textbf{Ventajas de usar Julia para Inventarios:}
		
		\begin{enumerate}[leftmargin=*,label=\textcolor{azulprincipal}{\arabic*.}]
			\item \textbf{Rendimiento Superior:} Julia ejecuta c\'alculos matem\'aticos a velocidad cercana a C
			\item \textbf{Sintaxis Clara:} C\'odigo f\'acil de leer y mantener
			\item \textbf{Ecosistema Cient\'ifico:} Paquetes especializados para optimizaci\'on
			\item \textbf{Interactividad:} Ideal para an\'alisis exploratorio y prototipado
			\item \textbf{Escalabilidad:} Desde problemas simples hasta sistemas complejos
		\end{enumerate}
		
		\textbf{Mejores Pr\'acticas Implementadas:}
		
		\begin{itemize}[leftmargin=*,label=\textcolor{verdeprincipal}{\faCheck}]
			\item Funciones modulares y reutilizables
			\item Documentaci\'on clara con docstrings
			\item Manejo de errores y validaci\'on de entrada
			\item An\'alisis de sensibilidad incorporado
			\item Visualizaci\'on de resultados
			\item Simulaci\'on Monte Carlo para incertidumbre
		\end{itemize}
		
	\end{tcolorbox}
	
	\begin{tcolorbox}[enhanced,colback=naranjaclaro,colframe=naranjaacento,boxrule=3pt,arc=12pt,
		drop shadow,title={\Large\bfseries\color{white} \faGlobe\ EXTENSIONES FUTURAS}]
		
		\textbf{Desarrollos Avanzados con Julia:}
		\begin{itemize}[leftmargin=*,label=\textcolor{azulprincipal}{\faArrowRight}]
			\item Optimizaci\'on multiobjetivo con JuMP.jl
			\item Modelos de inventario din\'amicos con DifferentialEquations.jl
			\item Machine Learning para predicci\'on de demanda con MLJ.jl
			\item Interfaces web interactivas con PlutoUI.jl
			\item Optimizaci\'on distribuida para problemas grandes
		\end{itemize}
		
	\end{tcolorbox}
	
	\begin{center}
		\begin{tcolorbox}[enhanced,colback=juliacolor,colframe=white,boxrule=2pt,arc=10pt,
			width=0.8\textwidth,center]
			\centering
			{\Large\bfseries\color{white} \faTrophy\ Julia: El futuro de la optimizaci\'on computacional en inventarios}
		\end{tcolorbox}
	\end{center}
	
\end{document}